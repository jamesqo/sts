In subsequent sections, I will implement algorithms for both dynamic and growth arrays. In this section, I define fields and properties for these data structures which the algorithms will use. \textbf{Fields} are variables associated with an object that may be read from or written to. \textbf{Properties} are trivial, constant-time methods that do not change state.

If $\VarList$ is a dynamic array, then it is assumed to have the following fields:

\begin{itemize}
	\item $\VarList.\FieldBuffer$ - The \textbf{buffer}, or raw array, that $\VarList$ stores its items in.
	\item $\VarList.\FieldSize$ - The number of items in $\VarList$.
\end{itemize}

As a dynamic array, $\VarList$ is also given the following properties. $:=$ denotes a definition, as opposed to $=$ which checks equivalence. Functions that return boolean values are suffixed with ?.

\begin{algorithm}
	\begin{algorithmic}
		\LineComment{Returns the capacity of $\VarList$}
		\State $\VarList.\FieldCapacity := \VarList.\FieldBuffer.\FieldLength$
		\State
		\LineComment{Returns whether $\VarList$ is full}
		\State $\VarList.\FieldFull := \VarList.\FieldSize = \VarList.\FieldCapacity$
	\end{algorithmic}
\end{algorithm}

When a dynamic array is instantiated, the following code should run:

\begin{algorithm}
	\begin{algorithmic}
		\Procedure{$\FuncConstructor$}{$\VarList$}
			\State $\VarList.\FieldBuffer \gets \FuncNewArray(0)$
			\State $\VarList.\FieldSize \gets 0$
		\EndProcedure
	\end{algorithmic}
\end{algorithm}

If $\VarList$ is a growth array, then it is assumed to have the following fields:

\begin{itemize}
	\item $\VarList.\FieldHead$ - The \textbf{head} of $\VarList$. It returns the buffer we are currently adding items to.
	\item $\VarList.\FieldTail$ - The \textbf{tail} of $\VarList$. It returns a dynamic array of references to buffers that have already been filled. This can be thought of as a two-dimensional array.\\
	{\HdrNote} It may seem strange for growth arrays to use dynamic arrays internally. There is little harm from the extra copying and allocations these dynamic arrays perform, however, since only $O(\log n)$ references are appended to them (proved in [ref3]).
	\item $\VarList.\FieldSize$ - The number of items in $\VarList$.
	\item $\VarList.\FieldCapacity$ - The \textbf{capacity} of $\VarList$. It returns the maximum number of items $\VarList$ can hold without resizing.
\end{itemize}

As a growth array, $\VarList$ is also given the following properties:

\begin{algorithm}
	\begin{algorithmic}
		\LineComment{Returns whether $\VarList$ is empty}
		\State $\VarList.\FieldEmpty := \VarList.\FieldSize = 0$
		\State
		\LineComment{Returns whether $\VarList$ is full}
		\State $\VarList.\FieldFull := \VarList.\FieldSize = \VarList.\FieldCapacity$
		\State
		\LineComment{Returns the capacity of $\FieldHead$}
		\State $\VarList.\FieldHeadCapacity := \VarList.\FieldHead.\FieldLength$
		\State
		\LineComment{Returns the size of $\FieldHead$}
		\LineComment{\textbf{Rationale:} $\FieldCapacity - \FieldHeadCapacity$ is the total capacity of the buffers in $\FieldTail$.}
		\LineComment{Then, $\FieldSize - \left( \FieldCapacity - \FieldHeadCapacity \right)$ is the number of items that were added}
		\LineComment{after depleting the buffers in $\FieldTail$.}
		\State $\VarList.\FieldHeadSize := \VarList.\FieldSize - \left(\VarList.\FieldCapacity - \VarList.\FieldHeadCapacity\right)$
	\end{algorithmic}
\end{algorithm}

When a growth array is instantiated, the following code should run:

\begin{algorithm}
	\begin{algorithmic}
		\Procedure{$\FuncConstructor$}{$\VarList$}
			\State $\VarList.\FieldHead \gets \FuncNewArray(0)$
			\State $\VarList.\FieldTail \gets \FuncNewDynamicArray()$
			\State $\VarList.\FieldSize \gets 0$
			\State $\VarList.\FieldCapacity \gets 0$
		\EndProcedure
	\end{algorithmic}
\end{algorithm}