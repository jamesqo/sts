In later sections, I will implement algorithms for both dynamic and growth arrays. In this section, I will define fields and properties for those algorithms to use. \textbf{Fields} are variables associated with an object that may be read from or written to. \textbf{Properties} are trivial, constant time methods that do not change state.

Let $\VarList$ be a dynamic array. I will give $\VarList$ the following fields:

\begin{itemize}
	\item $\VarList.\FieldBuffer$ -- The \textbf{buffer}, or raw array, that $\VarList$ stores its items in.
	\item $\VarList.\FieldSize$ -- The number of items in $\VarList$.
\end{itemize}

As a dynamic array, $\VarList$ is also given the following properties. ({\HdrNote} For pseudocode, I will use $:=$ to denote a definition, and $=$ to denote an equivalence check. Functions that return boolean values are suffixed with ?.)

\begin{algorithm}[H]
	\begin{algorithmic}
		\LineComment{Returns the capacity of $\VarList$}
		\State $\VarList.\FieldCapacity := \VarList.\FieldBuffer.\FieldLength$
		\State
		\LineComment{Returns whether $\VarList$ is full}
		\State $\VarList.\FieldFull := \VarList.\FieldSize = \VarList.\FieldCapacity$
	\end{algorithmic}
\end{algorithm}

The following code should run when $\VarList$ is instantiated:

\begin{algorithm}[H]
	\begin{algorithmic}
		\Procedure{$\FuncConstructor$}{$\VarList$}
			\State $\VarList.\FieldBuffer \gets \FuncNewArray(\VarInitCapacity)$
			\State $\VarList.\FieldSize \gets 0$
		\EndProcedure
	\end{algorithmic}
\end{algorithm}

Now, let $\VarList$ be a growth array. I will give it the following fields:

\begin{itemize}
	\item $\VarList.\FieldHead$ -- The \textbf{head} of $\VarList$. It returns the buffer new items are appended to.
	\item $\VarList.\FieldTail$ -- The \textbf{tail} of $\VarList$. (\textbf{Important note: The tail is a dynamic array.}) It returns a dynamic array of pointers to buffers that are already filled with items. The tail can be thought of as a two-dimensional array.\\
	({\HdrRemark} It may seem strange for a growth array to use the very data structure it is replacing. Lemma \ref{lem:VarUsefulIsOLogN} will show, however, that only $O(\log n)$ many pointers are appended to the tail. Thus, the extra copying and allocations the tail performs is minuscule compared to other work done by the growth array.)
	\item $\VarList.\FieldSize$ -- The number of items in $\VarList$.
	\item $\VarList.\FieldCapacity$ -- The \textbf{capacity} of $\VarList$. It returns the maximum number of items $\VarList$ can hold before it must grow.
\end{itemize}

As a growth array, $\VarList$ is also given the following properties:

\begin{algorithm}[H]
	\begin{algorithmic}
		\LineComment{Returns whether $\VarList$ is full}
		\State $\VarList.\FieldFull := \VarList.\FieldSize = \VarList.\FieldCapacity$
		\State
		\LineComment{Returns the \textbf{head capacity}, or the capacity of $\FieldHead$}
		\State $\VarList.\FieldHeadCapacity := \VarList.\FieldHead.\FieldLength$
		\State
		\LineComment{Returns the \textbf{head size}, or the size of $\FieldHead$}
		\LineComment{\textbf{Explanation:} $\FieldCapacity - \FieldHeadCapacity$ is the number of items in $\FieldTail$.}
		\LineComment{$\FieldSize - \left( \FieldCapacity - \FieldHeadCapacity \right)$ is the number of items not in $\FieldTail$ (meaning, in $\FieldHead$).}
		\State $\VarList.\FieldHeadSize := \VarList.\FieldSize - \left(\VarList.\FieldCapacity - \VarList.\FieldHeadCapacity\right)$
	\end{algorithmic}
\end{algorithm}

The following code should run when $\VarList$ is instantiated:

\begin{algorithm}[H]
	\begin{algorithmic}
		\Procedure{$\FuncConstructor$}{$\VarList$}
			\State $\VarList.\FieldHead \gets \FuncNewArray(\VarInitCapacity)$
			\State $\VarList.\FieldTail \gets \FuncNewDynamicArray()$
			\State $\VarList.\FieldSize \gets 0$
			\State $\VarList.\FieldCapacity \gets \VarInitCapacity$
		\EndProcedure
	\end{algorithmic}
\end{algorithm}