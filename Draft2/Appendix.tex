\appendix
\appendixpage

\section{Proofs of $\TextTilde$ Properties}

\subsection{Merging over Arithmetic Operations}
\label{pf:MergesOverArithmetic}

\begin{proof}[Proof (Theorem \ref{thm:MergesOverArithmeticAdd})]
	We will consider the statement about addition first. Recall from the note at \ref{subsec:AsymptoticProperties} that this statement is equivalent to:
	
	\textit{Let $\hat{f} \in P(f(n))$ and $\hat{g} \in P(g(n))$. Then $\hat{f} + \hat{g} \in P(f(n) + g(n))$.}
	
	We can show this by using the limit definition of $P$. For brevity, we will omit the parameter $n$ to each function. Since
	
	\begin{align*}
	\lim_{n \to \infty} \left( \frac{\hat{f} + \hat{g}}{f + g} \right) &= \lim_{n \to \infty} \left( \frac{\hat{f}}{f + g} \right) + \lim_{n \to \infty} \left( \frac{\hat{g}}{f + g} \right)\\
	&= \left( \lim_{n \to \infty} \frac{\hat{f}}{f} \right) \left( \lim_{n \to \infty} \left( \frac{f}{f + g} \right) \right) +\\
	&\ \ \ \ \left( \lim_{n \to \infty} \frac{\hat{g}}{g} \right) \left( \lim_{n \to \infty} \left( \frac{g}{f + g} \right) \right)\\
	&= \lim_{n \to \infty} \left( \frac{f}{f + g} \right) + \lim_{n \to \infty} \left( \frac{g}{f + g} \right)\\
	&= \lim_{n \to \infty} \left( \frac{f + g}{f + g} \right)\\
	&= 1
	\end{align*}
	
	it follows that $\hat{f} + \hat{g} \in P(f(n) + g(n))$. The statement about subtraction can be proved in a similar fashion.
\end{proof}

\begin{proof}[Proof (Theorem \ref{thm:MergesOverArithmeticMultiply})]
	We will consider the statement about multiplication first. Recall from the note at \ref{subsec:AsymptoticProperties} that this statement is equivalent to:
	
	\textit{Let $\hat{f} \in P(f(n))$ and $\hat{g} \in P(g(n))$. Then $\hat{f}\hat{g} \in P(f(n)g(n))$.}
	
	We can show this easily by multiplying the limits corresponding to $\hat{f}$ and $\hat{g}$:
	
	\begin{align*}
	\lim_{n \to \infty} \frac{\hat{f}(n)}{f(n)} &= 1\\
	\lim_{n \to \infty} \frac{\hat{g}(n)}{g(n)} &= 1\\
	\left( \lim_{n \to \infty} \frac{\hat{f}(n)}{f(n)} \right) \left( \lim_{n \to \infty} \frac{\hat{g}(n)}{g(n)} \right) &= 1 \cdot 1\\
	\lim_{n \to \infty} \frac{\hat{f}(n)\hat{g}(n)}{f(n)g(n)} &= 1
	\end{align*}
	
	it follows that $\hat{f}\hat{g} \in P(f(n)g(n))$. The statement about division can be proved by flipping the fraction for $\hat{g}$ before multiplying, resulting in
	
	\begin{align*}
	\lim_{n \to \infty} \frac{\hat{f}(n) / \hat{g}(n)}{f(n) / g(n)} = 1
	\end{align*}
	
	This implies $\hat{f} / \hat{g} \in P(f(n) / g(n))$.
\end{proof}

\subsection{Transitivity}
\label{pf:Transitivity}

\begin{theorem}
	If $f \TextTilde g$ and $g \TextTilde h$, then $f \TextTilde h$.
\end{theorem}

\begin{proof}
	By definition, $\lim_{n \to \infty} g / f = 1$ and $\lim_{n \to \infty} h / g = 1$. Multiplying the two equations, $\lim_{n \to \infty} h / f = 1$ which implies $f \TextTilde h$.
\end{proof}

\subsection{Removal of Lower-Order Terms}
\label{pf:RemovesLowerOrderTerms}

\begin{theorem}
	If $\lim_{n \to \infty} \frac{g(n)}{f(n)} = 0$, then $P(f(n) + g(n)) = P(f(n))$.
\end{theorem}

\begin{proof}
	\begin{align*}
	\lim_{n \to \infty} \left( \frac{f(n) + g(n)}{f(n)} \right) &= \lim_{n \to \infty} \frac{f(n)}{f(n)} + \lim_{n \to \infty} \frac{g(n)}{f(n)}\\
	&= 1 + 0\\
	&= 1
	\end{align*}
	
	Then $f(n) + g(n) \in P(f(n))$. Since it is also true that $f(n) + g(n) \in P(f(n) + g(n))$, it follows from \ref{pf:Transitivity} that $P(f(n)) = P(f(n) + g(n))$.
\end{proof}