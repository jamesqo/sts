\appendix
\appendixpage

\section{Proofs of $\sim$ Properties}

\subsection{$\sim$ is an Equivalence Relation}

\begin{proof}[Proof (Theorem \ref{thm:EquivalenceRelation})]
	Clearly, $\ExprNToInfty \dfrac{f}{f} = 1$, so $f \sim f$. Thus, $\sim$ is reflexive.
	
	Suppose $f \sim g$. Then $\ExprNToInfty \dfrac{f}{g} = 1$. Since both $\ExprNToInfty 1$ and $\ExprNToInfty \dfrac{f}{g}$ exist and the latter is nonzero, $\dfrac{\ExprNToInfty 1}{\ExprNToInfty (f / g)}$ = $\ExprNToInfty \dfrac{g}{f}$. Since the left-hand side evaluates to $1$, $g \sim f$. Thus, $\sim$ is symmetric.
	
	Suppose $f \sim g$ and $g \sim h$. By definition, $\ExprNToInfty \dfrac{f}{g} = 1$ and $\ExprNToInfty \dfrac{g}{h} = 1$. Since both limits exist, their product is $\ExprNToInfty \left( \dfrac{f}{g} \cdot \dfrac{g}{h} \right) = \ExprNToInfty \dfrac{f}{h}$. Since this product is $1$, $f \sim h$. Thus, $\sim$ is transitive.
\end{proof}

\subsection{$\sim$ Removes Lower-Order Terms}

\begin{proof}[Proof (Theorem \ref{thm:RemovesLowerOrderTerms})]
	Let $g = o(f)$. By definition, $\ExprNToInfty \dfrac{g}{f} = 0$. Since $\ExprNToInfty \dfrac{f}{f}$ and $\ExprNToInfty \dfrac{g}{f}$ both exist,
	\begin{align*}
	\ExprNToInfty \dfrac{f + g}{f} = \ExprNToInfty \dfrac{f}{f} + \ExprNToInfty \dfrac{g}{f} = 1 + 0 = 1
	\end{align*}
	It follows that $f + g = f + o(f) \sim f$.
\end{proof}

\subsubsection{$f + c \sim f$ for Unbounded $f$}

\begin{proof}[Proof (Corollary \ref{coro:PlusConstant})]
	If $f$ is unbounded, then $c = o(f)$. Applying Theorem \ref{thm:RemovesLowerOrderTerms}, $f + c \sim f$.
\end{proof}

\subsection{$\sim$ Respects $+$, $\times$, and $\div$}

\begin{proof}[Proof (Theorem \ref{thm:MergesOverOps})]
	Let $\FuncQuotient = \dfrac{f + g}{\Tweak{f} + \Tweak{g}}$, and let $\FuncDenom = \Tweak{g} + \Tweak{g}^2 / \Tweak{f}$. $\FuncQuotient$ may be expressed as
	\begin{align*}
	\FuncQuotient &= \frac{(\Tweak{g} / \Tweak{f})(f + g)}{(\Tweak{g} / \Tweak{f})(\Tweak{f} + \Tweak{g})}\\
	&= \frac{\Tweak{g}(f / \Tweak{f}) + \Tweak{g}g / \Tweak{f}}{\FuncDenom}\\
	&= \frac{\Tweak{g}(f / \Tweak{f}) + (\Tweak{g}^2 / \Tweak{f})(g / \Tweak{g})}{\FuncDenom}\\
	&= \frac{\Tweak{g}}{d} \cdot \frac{f}{\Tweak{f}} + \frac{\Tweak{g}^2 / \Tweak{f}}{\FuncDenom} \cdot \frac{g}{\Tweak{g}}
	\end{align*}
	Now, consider that
	\begin{align*}
	\FuncQuotient - 1 &= \FuncQuotient - \frac{\FuncDenom}{\FuncDenom} = \FuncQuotient - \frac{\Tweak{g}}{\FuncDenom} - \frac{\Tweak{g}^2 / \Tweak{f}}{\FuncDenom}\\
	&= \frac{\Tweak{g}}{d} \cdot \frac{f}{\Tweak{f}} + \frac{\Tweak{g}^2 / \Tweak{f}}{\FuncDenom} \cdot \frac{g}{\Tweak{g}} - \frac{\Tweak{g}}{\FuncDenom} - \frac{\Tweak{g}^2 / \Tweak{f}}{\FuncDenom}\\
	&= \frac{\Tweak{g}}{d} \cdot \left( \frac{f}{\Tweak{f}} - 1 \right) + \frac{\Tweak{g}^2 / \Tweak{f}}{\FuncDenom} \cdot \left( \frac{g}{\Tweak{g}} - 1 \right)\\
	\ExprNToInfty \left( \FuncQuotient - 1 \right) &= \ExprNToInfty \left( \frac{\Tweak{g}}{d} \cdot \left( \frac{f}{\Tweak{f}} - 1 \right) + \frac{\Tweak{g}^2 / \Tweak{f}}{\FuncDenom} \cdot \left( \frac{g}{\Tweak{g}} - 1 \right) \right)\\
	&= \ExprNToInfty \left( \frac{\Tweak{g}}{d} \cdot \left( \frac{f}{\Tweak{f}} - 1 \right) \right) + \ExprNToInfty \left( \frac{\Tweak{g}^2 / \Tweak{f}}{\FuncDenom} \cdot \left( \frac{g}{\Tweak{g}} - 1 \right) \right)
	\end{align*}
	Since $\Tweak{g}$ and $\Tweak{g}^2 / \Tweak{f}$ sum to $\FuncDenom$ and all functions are positive, $\dfrac{\Tweak{g}}{\FuncDenom}$ and $\dfrac{\Tweak{g}^2 / \Tweak{f}}{\FuncDenom}$ are bounded between $0$ and $1$ as $n \to \infty$. From the given, $\ExprNToInfty \left( \dfrac{f}{\Tweak{f}} - 1 \right)$ and $\ExprNToInfty \left( \dfrac{g}{\Tweak{g}} - 1 \right)$ both exist and equal $0$. The limit of a bounded expression times one approaching $0$ is $0$; thus, both limits on the right-hand side are $0$.
	
	Substituting, I receive $\ExprNToInfty (\FuncQuotient - 1) = 0$, so $\ExprNToInfty \FuncQuotient = 1$. Since $\FuncQuotient$ is defined as $\dfrac{f + g}{\Tweak{f} + \Tweak{g}}$, this shows that $f + g \sim \Tweak{f} + \Tweak{g}$, proving the theorem statement for addition.
	
	The statement is proven much more easily for multiplication. Since both $\ExprNToInfty \dfrac{f}{\Tweak{f}}$ and $\ExprNToInfty \dfrac{g}{\Tweak{g}}$ exist,
	\begin{align*}
	\ExprNToInfty \frac{fg}{\Tweak{f}\Tweak{g}} = \left( \ExprNToInfty \frac{f}{\Tweak{f}} \right) \left( \ExprNToInfty \frac{g}{\Tweak{g}} \right) = 1 \cdot 1 = 1
	\end{align*}
	It follows that $fg \sim \Tweak{f}\Tweak{g}$. If the limit for $g$ is flipped before multiplying, the following results:
	\begin{align*}
	\ExprNToInfty \frac{f / g}{\Tweak{f} / \Tweak{g}} = 1
	\end{align*}
	This implies that $\dfrac f g \sim \dfrac {\Tweak{f}} {\Tweak{g}}$.
\end{proof}

\subsubsection{$\sim$ Equations can be Algebraically Manipulated}

\begin{proof}[Proof (Corollary \ref{coro:BothSides})]
	This immediately follows from Theorem \ref{thm:MergesOverOps} by taking $\Tweak{g} = g$.
\end{proof}

\subsection{Asymptotic Functions Have the Same Big-O Class}

\begin{proof}[Proof (Theorem \ref{thm:SameBigOClass})]
	Let $r = \dfrac{f}{g}$. Since $\ExprNToInfty r = 1$, $r \leq 2$ for sufficiently large $n$. By definition, $\exists c : g \leq c h$ for sufficiently large $n$, where $c$ is a positive constant. Then, $f = rg \leq rc h \leq 2c h$ for sufficiently large $n$. Thus, $f = O(h)$.
\end{proof}

\subsection{Asymptotic Functions may be Interchanged in $\FluteLess$ and $\FluteLeq$ Inequalities}

\begin{proof}[Proof (Theorem \ref{thm:InterchangeableInInequality})]
	By definition, $\ExprNToInfty \dfrac{f}{\Tweak{f}} = 1$ and $\ExprNToInfty \dfrac{\Tweak{f}}{\TweakSecond{f}} < 1$. Since both limits exist,
	\begin{align*}
	\ExprNToInfty \frac{f}{\TweakSecond{f}} = \left( \ExprNToInfty \frac{f}{\Tweak{f}} \right) \left( \ExprNToInfty \frac{\Tweak{f}}{\TweakSecond{f}} \right) < 1 \cdot 1 = 1
	\end{align*}
	It follows that $f \FluteLess \TweakSecond{f}$.
	
	The second statement may be proved in a similar manner. Since $\sim$ is symmetric, $\TweakSecond{g} \sim g$. Applying the same argument as before,
	\begin{align*}
	\ExprNToInfty \frac{\Tweak{g}}{g} = \left( \ExprNToInfty \frac{\Tweak{g}}{\TweakSecond{g}} \right) \left( \ExprNToInfty \frac{\TweakSecond{g}}{g} \right) < 1 \cdot 1 = 1
	\end{align*}
	Thus, $\Tweak{g} \FluteLess g$.
	
	From these results and the transitivity of $\sim$, it can trivially be shown that the same statements are true when $\FluteLess$ is replaced with $\FluteLeq$.
\end{proof}

\subsection{$\FluteLess$ and $\FluteLeq$ Inequalities can be Algebraically Manipulated}

\begin{proof}[Proof (Theorem \ref{thm:BothSidesInequality})]
	Suppose that $f \FluteLess g$. By definition, $\ExprNToInfty \dfrac{f}{g} < 1$. For any function $h$, $\ExprNToInfty \dfrac{fh}{gh} < 1$ and $\ExprNToInfty \dfrac{f / h}{g / h} < 1$, so respectively $fh \FluteLess gh$ and $\dfrac{f}{h} \FluteLess \dfrac{g}{h}$.
	
	From these results and Corollary \ref{coro:BothSides}, it can trivially be shown that the same statement is true when $\FluteLess$ is replaced with $\FluteLeq$.
\end{proof}

\subsection{Same-Order Terms may be Added to Both Sides of $\FluteLess$ and $\FluteLeq$ Inequalities}

\begin{proof}[Proof (Theorem \ref{thm:BothSidesInequalityAdd})]
	If $\dfrac{f}{h}$ does not approach $0$ as $n \to \infty$, then it is bounded below by some positive constant $c_1$, and $\ExprNToInfty \dfrac{c_1 h}{f} \leq 1$. From $f \FluteLess g$, it follows that $\ExprNToInfty \dfrac{f}{g} < 1$. Since $g = O(h)$, $\exists c_2 : \ExprNToInfty \dfrac{g}{c_2 h} \leq 1$. Multiplying these limits, I receive
	\begin{align*}
	\left( \ExprNToInfty \frac{c_1 h}{f} \right) \left( \ExprNToInfty \dfrac{f}{g} \right) \left( \ExprNToInfty \dfrac{g}{c_2 h} \right) = \ExprNToInfty \frac{c_1 h}{c_2 h} = \frac{c_1}{c_2}
	\end{align*}
	Since at least one limit is less than $1$ and no limit exceeds $1$, the product, $\dfrac{c_1}{c_2}$, is less than $1$. Now, consider that
	\begin{align*}
	\ExprNToInfty \frac{f + h}{g + h} \leq \ExprNToInfty \frac{c_1 h + h}{c_2 h + h} = \frac{c_1 + 1}{c_2 + 1}
	\end{align*}
	Since $c_1 < c_2$, $c_1 + 1 < c_2 + 1$ so $\dfrac{c_1 + 1}{c_2 + 1} < 1$. Thus, $f + h \FluteLess g + h$.
	
	From these results and Corollary \ref{coro:BothSides}, it can trivially be shown that the same statement is true when $\FluteLess$ is replaced with $\FluteLeq$.
\end{proof}