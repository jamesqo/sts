% Todo: Name 'Appendix', not 'Appendices'
\appendix
\appendixpage

\section{Proofs of $\sim$ Properties}

\subsection{$\sim$ is an Equivalence Relation}

\begin{proof}[Proof (Theorem \ref{thm:EquivalenceRelation})]
	Clearly, $\ExprNToInfty \dfrac{f}{f} = 1$, so $f \sim f$. Thus, $\sim$ is reflexive.
	
	Suppose $f \sim g$. Then $\ExprNToInfty \dfrac{f}{g} = 1$. Since both $\ExprNToInfty 1$ and $\ExprNToInfty \dfrac{f}{g}$ exist and the latter is nonzero, $\dfrac{\ExprNToInfty 1}{\ExprNToInfty (f / g)}$ = $\ExprNToInfty \dfrac{g}{f}$. Since the left-hand side evaluates to $1$, $g \sim f$. Thus, $\sim$ is symmetric.
	
	Suppose $f \sim g$ and $g \sim h$. By definition, $\ExprNToInfty \dfrac{f}{g} = 1$ and $\ExprNToInfty \dfrac{g}{h} = 1$. Since both limits exist, their product is $\ExprNToInfty \left( \dfrac{f}{g} \cdot \dfrac{g}{h} \right) = \ExprNToInfty \dfrac{f}{h}$. Since this product is $1$, $f \sim h$. Thus, $\sim$ is transitive.
\end{proof}

\subsection{$\sim$ Removes Lower-Order Terms}

\begin{proof}[Proof (Theorem \ref{thm:RemovesLowerOrderTerms})]
	Let $h = O(g)$. By definition, $h \leq cg$ for sufficiently large $n$, where $c$ is a positive constant. Dividing both sides by $f$ and taking the limit, $\ExprNToInfty \dfrac{h}{f} \leq \ExprNToInfty c \left( \dfrac{g}{f} \right)$. Since $\ExprNToInfty \dfrac{g}{f}$ exists and equals $0$, the right-hand side equals $c \cdot 0 = 0$. However, $\ExprNToInfty \dfrac{h}{f} \geq 0$ since both $h$ and $f$ are positive. By the squeeze theorem, $\ExprNToInfty \dfrac{h}{f} = 0$.
	
	Since $\ExprNToInfty \dfrac{f}{f}$ and $\ExprNToInfty \dfrac{h}{f}$ both exist,
	\begin{align*}
	\ExprNToInfty \dfrac{f + h}{f} = \ExprNToInfty \dfrac{f}{f} + \ExprNToInfty \dfrac{h}{f} = 1 + 0 = 1
	\end{align*}
	It follows that $f + h = f + O(g) \sim f$.
\end{proof}

\subsubsection{$f + c \sim f$ for Unbounded $f$}

\begin{proof}[Proof (Corollary \ref{coro:PlusConstant})]
	If $f$ is unbounded, $\ExprNToInfty \dfrac{c}{f} = 0$. Applying Theorem \ref{thm:RemovesLowerOrderTerms}, $f + c \sim f$.
\end{proof}

\subsection{$\sim$ Merges over $+$, $\times$, and $\div$}

\begin{proof}[Proof (Theorem \ref{thm:MergesOverOps})]
	% TODO: Prove statement for addition.
	
	The statement about multiplication is more easily proven. Since both $\ExprNToInfty \dfrac{f}{\Tweak{f}}$ and $\ExprNToInfty \dfrac{g}{\Tweak{g}}$ exist,
	\begin{align*}
	\ExprNToInfty \frac{fg}{\Tweak{f}\Tweak{g}} = \left( \ExprNToInfty \frac{f}{\Tweak{f}} \right) \left( \ExprNToInfty \frac{g}{\Tweak{g}} \right) = 1 \cdot 1 = 1
	\end{align*}
	It follows that $fg \sim \Tweak{f}\Tweak{g}$. If the limit for $g$ is flipped before multiplying, the following results:
	\begin{align*}
	\ExprNToInfty \frac{f / g}{\Tweak{f} / \Tweak{g}} = 1
	\end{align*}
	This shows that $\dfrac f g \sim \dfrac {\Tweak{f}} {\Tweak{g}}$.
\end{proof}

\subsubsection{$\sim$ Relations can be Algebraically Manipulated}

\begin{proof}[Proof (Corollary \ref{coro:BothSides})]
	Since $\sim$ is reflexive, $g \sim g$. Taking $\Tweak{g} = g$ for Theorem \ref{thm:MergesOverOps}, the corollary statement follows.
\end{proof}

\subsection{Asymptotic Functions Have the Same Big-O Class}

\begin{proof}[Proof (Theorem \ref{thm:SameBigOClass})]
	This will be a proof by contradiction. Assume $f = O(\Tweak{f})$ where $O(\Tweak{f}) \neq O(\Tweak{g})$. Then $\dfrac {O(\Tweak{f})} {O(\Tweak{g})} \neq 1$. However, $\ExprNToInfty \dfrac{f}{g} = \dfrac {O(\Tweak{f})} {O(\Tweak{g})}$, so $\ExprNToInfty \dfrac{f}{g} \neq 1$. It follows that $f \not\sim g$, which contradicts the given. Thus, it must be true that $f = O(\Tweak{g})$.
\end{proof}

\subsection{Asymptotic Functions may be Swapped in Strict Asymptotic Inequalities}

\begin{proof}[Proof (Theorem \ref{thm:InterchangeableInInequality})]
	By definition, $\ExprNToInfty \dfrac{f}{\Tweak{f}} = 1$ and $\ExprNToInfty \dfrac{\Tweak{f}}{\TweakSecond{f}} < 1$. Since both limits exist,
	\begin{align*}
	\ExprNToInfty \frac{f}{\TweakSecond{f}} = \left( \ExprNToInfty \frac{f}{\Tweak{f}} \right) \left( \ExprNToInfty \frac{\Tweak{f}}{\TweakSecond{f}} \right) < 1 \cdot 1 = 1
	\end{align*}
	It follows that $f \FluteLess \TweakSecond{f}$.
	
	The second statement may be proved in a similar manner. Since $\sim$ is symmetric, $\TweakSecond{g} \sim g$. Applying the same argument as before,
	\begin{align*}
	\ExprNToInfty \frac{\Tweak{g}}{g} = \left( \ExprNToInfty \frac{\Tweak{g}}{\TweakSecond{g}} \right) \left( \ExprNToInfty \frac{\TweakSecond{g}}{g} \right) < 1 \cdot 1 = 1
	\end{align*}
	Thus, $\Tweak{g} \FluteLess g$.
\end{proof}

\subsubsection{Asymptotic Functions may be Swapped in Non-Strict Asymptotic Inequalities}

\begin{proof}[Proof (Corollary \ref{coro:InterchangeableInNonStrictInequality})]
	Suppose $f \sim \Tweak{f}$ and $\Tweak{f} \FluteLeq \TweakSecond{f}$. By definition, $\Tweak{f} \FluteLess \TweakSecond{f}$ or $\Tweak{f} \sim \TweakSecond{f}$. In the first case, from Theorem \ref{thm:InterchangeableInInequality} $f \FluteLess \TweakSecond{f}$. In the second case, since $\sim$ is transitive $f \sim \TweakSecond{f}$. In both cases, it is true that $f \FluteLeq \TweakSecond{f}$.
	
	The second statement is, again, proved in a similar manner. If $\Tweak{g} \FluteLeq \TweakSecond{g}$, then $\Tweak{g} \FluteLess \TweakSecond{g}$ or $\Tweak{g} \sim \TweakSecond{g}$. In the first case, $\Tweak{g} \FluteLess g$; in the second, $\Tweak{g} \sim g$. In both cases, $\Tweak{g} \FluteLeq g$.
\end{proof}

\subsubsection{$\FluteLess$ and $\FluteLeq$ Relations can be Algebraically Manipulated}

\begin{proof}[Proof (Theorem \ref{thm:BothSidesInequality})]
	Suppose that $f \FluteLess g$. By definition, $\ExprNToInfty \dfrac{f}{g} < 1$. For any function $h$, $\ExprNToInfty \dfrac{fh}{gh} < 1$ and $\ExprNToInfty \dfrac{f / h}{g / h} < 1$, so respectively $fh \FluteLess gh$ and $\dfrac{f}{h} \FluteLess \dfrac{g}{h}$.
	
	If I suppose $f \FluteLeq g$, then $f \FluteLess g$ or $f \sim g$. The first statement implies $fh \FluteLess gh$, and the second implies $fh \sim gh$ (from Corollary \ref{coro:BothSides}). In either case, $fh \FluteLeq gh$. Using the same logic, it is true that $\dfrac{f}{h} \FluteLeq \dfrac{g}{h}$.
\end{proof}