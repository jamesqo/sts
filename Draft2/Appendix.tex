\appendix
\appendixpage

\section{Proving $\sim$ Properties}

\subsection{$\sim$ Merges over $+$, $\times$, and $\div$}
\label{pf:MergesOverArithmetic}
\begin{proof}[Proof (Theorem \ref{thm:MergesOverOps})]
	The statement about multiplication can be shown easily by multiplying the limits corresponding to $f$ and $g$:
	\begin{align*}
	\ExprNToInfty \frac{f}{\Tweak{f}} &= 1\\
	\ExprNToInfty \frac{g}{\Tweak{g}} &= 1\\
	\left( \ExprNToInfty \frac{f}{\Tweak{f}} \right) \left( \ExprNToInfty \frac{g}{\Tweak{g}} \right) &= 1 \cdot 1\\
	\ExprNToInfty \frac{fg}{\Tweak{f}\Tweak{g}} &= 1
	\end{align*}
	it follows that $fg \sim \Tweak{f}\Tweak{g}$. The statement about division can be proved by flipping the limit for $g$ before multiplying, resulting in
	\begin{align*}
	\ExprNToInfty \frac{f / g}{\Tweak{f} / \Tweak{g}} = 1
	\end{align*}
	This shows $\dfrac f g \sim \dfrac {\Tweak{f}} {\Tweak{g}}$.
\end{proof}

\subsection{Removal of Lower-Order Terms}
\label{pf:RemovesLowerOrderTerms}

\begin{proof}[Proof (Theorem \ref{thm:RemovesLowerOrderTerms})]
	Since
	
	\begin{align*}
	\ExprNToInfty \left( \frac{f + g}{f} \right) &= \ExprNToInfty \frac{f}{f} + \ExprNToInfty \frac{g}{f}\\
	&= 1 + 0\\
	&= 1
	\end{align*}
	
	it follows that $f + g \sim f$.
\end{proof}

\subsection{Transitivity}
\label{pf:Transitivity}

\begin{proof}[Proof (Theorem \ref{thm:Transitivity})]
By definition, $\ExprNToInfty g / f = 1$ and $\ExprNToInfty h / g = 1$. Multiplying the two equations, $\ExprNToInfty h / f = 1$ which implies $f \sim h$.
\end{proof}