\appendix
\appendixpage

\section{Proofs of Properties of Big-P Notation}

\subsection{Merging over Arithmetic Operations}
\label{MergingOverArithmetic}

\begin{theorem}
	$P$ merges over addition and subtraction. That is,
	\begin{align*}
	P(f(n)) + P(g(n)) &= P(f(n) + g(n))\\
	P(f(n)) - P(g(n)) &= P(f(n) - g(n))
	\end{align*}
\end{theorem}

\begin{proof}
	We will consider the statement about addition first. Recall from the note at \ref{BigPProperties} that this statement is equivalent to:
	
	\textit{Let $f' \in P(f(n))$ and $g' \in P(g(n))$. Then $f' + g' \in P(f(n) + g(n))$.}
	
	We can show this by using the limit definition of $P$. For brevity, we will omit the parameter $n$ to each function. Since
	
	\begin{align*}
	\lim_{n \to \infty} \left( \frac{f' + g'}{f + g} \right) &= \lim_{n \to \infty} \left( \frac{f'}{f + g} \right) + \lim_{n \to \infty} \left( \frac{g'}{f + g} \right)\\
	&= \left( \lim_{n \to \infty} \frac{f'}{f} \right) \left( \lim_{n \to \infty} \left( \frac{f}{f + g} \right) \right) +\\
	&\ \ \ \ \left( \lim_{n \to \infty} \frac{g'}{g} \right) \left( \lim_{n \to \infty} \left( \frac{g}{f + g} \right) \right)\\
	&= \lim_{n \to \infty} \left( \frac{f}{f + g} \right) + \lim_{n \to \infty} \left( \frac{g}{f + g} \right)\\
	&= \lim_{n \to \infty} \left( \frac{f + g}{f + g} \right)\\
	&= 1
	\end{align*}
	
	it follows that $f' + g' \in P(f(n) + g(n))$. The statement about subtraction can be proved in a similar fashion.
\end{proof}

\begin{theorem}
	$P$ merges over multiplication and division. That is,
	\begin{align*}
	P(f(n))P(g(n)) &= P(f(n)g(n))\\
	\frac {P(f(n))} {P(g(n))} &= P\left(\frac{f(n)} {g(n)}\right)
	\end{align*}
\end{theorem}

\begin{proof}
	We will consider the statement about multiplication first. Recall from the note at \ref{BigPProperties} that this statement is equivalent to:
	
	\textit{Let $f' \in P(f(n))$ and $g' \in P(g(n))$. Then $f'g' \in P(f(n)g(n))$.}
	
	We can show this easily by multiplying the limits corresponding to $f'$ and $g'$:
	
	\begin{align*}
	\lim_{n \to \infty} \frac{f'(n)}{f(n)} &= 1\\
	\lim_{n \to \infty} \frac{g'(n)}{g(n)} &= 1\\
	\left( \lim_{n \to \infty} \frac{f'(n)}{f(n)} \right) \left( \lim_{n \to \infty} \frac{g'(n)}{g(n)} \right) &= 1 \cdot 1\\
	\lim_{n \to \infty} \frac{f'(n)g'(n)}{f(n)g(n)} &= 1
	\end{align*}
	
	it follows that $f'g' \in P(f(n)g(n))$. The statement about division can be proved by flipping the fraction for $g'$ before multiplying, resulting in
	
	\begin{align*}
	\lim_{n \to \infty} \frac{f'(n) / g'(n)}{f(n) / g(n)} = 1
	\end{align*}
	
	This implies $f' / g' \in P(f(n) / g(n))$.
\end{proof}

\subsection{Unique Class Lemma}
\label{UniqueClassOfFunctions}

\begin{lemma}
	Each function $f(n)$ has a unique class of functions under $P$. That is, suppose $f \in P(g(n))$ and $f \in P(h(n))$. Then $P(g(n)) = P(h(n))$.
\end{lemma}

\begin{proof}
	By definition, $\lim_{n \to \infty} \frac{f(n)}{g(n)} = 1$ and $\lim_{n \to \infty} \frac{f(n)}{h(n)} = 1$. Flipping the first equation and multiplying, we receive
	
	\begin{align*}
	\left( \lim_{n \to \infty} \frac {g(n)} {f(n)} \right) \left( \lim_{n \to \infty} \frac {f(n)} {h(n)} \right) &= 1\\
	\lim_{n \to \infty} \frac {g(n)} {h(n)} &= 1\\
	g &\in P(h(n))
	\end{align*}
	
	Now, consider any $g' \in P(g(n))$. Since
	
	\begin{align*}
	\lim_{n \to \infty} \frac {g'(n)} {g(n)} &= 1\\
	\left( \lim_{n \to \infty} \frac {g'(n)} {g(n)} \right) \left( \lim_{n \to \infty} \frac{g(n)}{h(n)} \right) &= 1\\
	\lim_{n \to \infty} \frac{g'(n)}{h(n)} &= 1
	\end{align*}
	
	it follows that $g' \in P(h(n))$, and thus $P(g(n)) \subseteq P(h(n))$. By a similar argument, it can be shown $P(h(n)) \subseteq P(g(n))$. Thus it must be true that $P(g(n)) = P(h(n))$.
\end{proof}

\subsection{Removal of Lower-Order Terms}
\label{ScrubsLowerOrderTerms}

\begin{theorem}
	If $\lim_{n \to \infty} \frac{g(n)}{f(n)} = 0$, then $P(f(n) + g(n)) = P(f(n))$.
\end{theorem}

\begin{proof}
	\begin{align*}
	\lim_{n \to \infty} \left( \frac{f(n) + g(n)}{f(n)} \right) &= \lim_{n \to \infty} \frac{f(n)}{f(n)} + \lim_{n \to \infty} \frac{g(n)}{f(n)}\\
	&= 1 + 0\\
	&= 1
	\end{align*}
	
	Then $f(n) + g(n) \in P(f(n))$. Since it is also true that $f(n) + g(n) \in P(f(n) + g(n))$, it follows from \ref{UniqueClassOfFunctions} that $P(f(n)) = P(f(n) + g(n))$.
\end{proof}