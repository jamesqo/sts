\appendix
\appendixpage

\section{Proofs of $\TextTilde$ Properties}

\subsection{Merging over Arithmetic Operations}
\label{pf:MergesOverArithmetic}

\begin{proof}[Proof (Theorem \ref{thm:MergesOverArithmeticAdd})]
\end{proof}

\begin{proof}[Proof (Theorem \ref{thm:MergesOverArithmeticMultiply})]
	This can be easily shown by multiplying the limits corresponding to $\hat{f}$ and $\hat{g}$:
	
	\begin{align*}
	\lim_{n \to \infty} \frac{\hat{f}}{f} &= 1\\
	\lim_{n \to \infty} \frac{\hat{g}}{g} &= 1\\
	\left( \lim_{n \to \infty} \frac{\hat{f}}{f} \right) \left( \lim_{n \to \infty} \frac{\hat{g}}{g} \right) &= 1 \cdot 1\\
	\lim_{n \to \infty} \frac{\hat{f}\hat{g}}{fg} &= 1
	\end{align*}
	
	it follows that $\hat{f}\hat{g} \TextTilde fg$. The statement about division can be proved by flipping the fraction for $\hat{g}$ before multiplying, resulting in
	
	\begin{align*}
	\lim_{n \to \infty} \frac{\hat{f} / \hat{g}}{f / g} = 1
	\end{align*}
	
	This shows $\hat{f} / \hat{g} \TextTilde f / g$.
\end{proof}

\subsection{Removal of Lower-Order Terms}
\label{pf:RemovesLowerOrderTerms}

\begin{proof}[Proof (Theorem \ref{thm:RemovesLowerOrderTerms})]
	Since
	
	\begin{align*}
	\lim_{n \to \infty} \left( \frac{f + g}{f} \right) &= \lim_{n \to \infty} \frac{f}{f} + \lim_{n \to \infty} \frac{g}{f}\\
	&= 1 + 0\\
	&= 1
	\end{align*}
	
	it follows that $f + g \TextTilde f$.
\end{proof}

\subsection{Transitivity}
\label{pf:Transitivity}

\begin{proof}[Proof (Theorem \ref{thm:Transitivity})]
By definition, $\lim_{n \to \infty} g / f = 1$ and $\lim_{n \to \infty} h / g = 1$. Multiplying the two equations, $\lim_{n \to \infty} h / f = 1$ which implies $f \TextTilde h$.
\end{proof}