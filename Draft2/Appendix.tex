% Todo: Name 'Appendix', not 'Appendices'
\appendix
\appendixpage

\section{Proofs of $\sim$ Properties}

\subsection{$\sim$ is an Equivalence Relation}

\begin{proof}[Proof (Theorem \ref{thm:EquivalenceRelation})]
	Clearly, $\ExprNToInfty \dfrac{f}{f} = 1$, so $f \sim f$. Thus, $\sim$ is reflexive.
	
	Suppose $f \sim g$. Then $\ExprNToInfty \dfrac{f}{g} = 1$. Since both $\ExprNToInfty 1$ and $\ExprNToInfty \dfrac{f}{g}$ exist and the latter is nonzero, $\dfrac{\ExprNToInfty 1}{\ExprNToInfty (f / g)}$ = $\ExprNToInfty \dfrac{g}{f}$. Since the left-hand side evaluates to $1$, $g \sim f$. Thus, $\sim$ is symmetric.
	
	Suppose $f \sim g$ and $g \sim h$. By definition, $\ExprNToInfty \dfrac{f}{g} = 1$ and $\ExprNToInfty \dfrac{g}{h} = 1$. Since both limits exist, their product is $\ExprNToInfty \left( \dfrac{f}{g} \cdot \dfrac{g}{h} \right) = \ExprNToInfty \dfrac{f}{h}$. Since this product is $1$, $f \sim h$. Thus, $\sim$ is transitive.
\end{proof}

\subsection{$\sim$ Removes Lower-Order Terms}

\begin{proof}[Proof (Theorem \ref{thm:RemovesLowerOrderTerms})]
	Since
	\begin{align*}
	\ExprNToInfty \frac{f + g}{f} &= \ExprNToInfty \frac{f}{f} + \ExprNToInfty \frac{g}{f}\\
	&= 1 + 0\\
	&= 1
	\end{align*}
	it follows that $f + g \sim f$.
\end{proof}

\subsection{$\sim$ Merges over $+$, $\times$, and $\div$}

\begin{proof}[Proof (Theorem \ref{thm:MergesOverOps})]
	% TODO: Prove statement for addition.
	
	The statement about multiplication can be shown easily by multiplying the limits corresponding to $f$ and $g$:
	\begin{align*}
	\ExprNToInfty \frac{f}{\Tweak{f}} &= 1\\
	\ExprNToInfty \frac{g}{\Tweak{g}} &= 1\\
	\left( \ExprNToInfty \frac{f}{\Tweak{f}} \right) \left( \ExprNToInfty \frac{g}{\Tweak{g}} \right) &= 1 \cdot 1\\
	\ExprNToInfty \frac{fg}{\Tweak{f}\Tweak{g}} &= 1
	\end{align*}
	it follows that $fg \sim \Tweak{f}\Tweak{g}$. The statement about division can be proved by flipping the limit for $g$ before multiplying, resulting in
	\begin{align*}
	\ExprNToInfty \frac{f / g}{\Tweak{f} / \Tweak{g}} = 1
	\end{align*}
	This shows $\dfrac f g \sim \dfrac {\Tweak{f}} {\Tweak{g}}$.
\end{proof}