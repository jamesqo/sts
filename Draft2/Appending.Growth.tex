\HdrGrowthArrayImpl

In this section, $\VarList$ is a growth array. The following algorithm implements appending for growth arrays.

\begin{algorithm}
	\begin{algorithmic}[1]
		\Procedure{$\FuncAppend$}{$\VarList, item$}
		\If{$\VarList.\FieldFull$}
		\State $\VarList.\FuncGrow()$
		\EndIf
		
		\State $\VarList.\FieldHead[\VarList.\FieldHeadSize] \gets item$
		\State $\VarList.\FieldSize \gets \VarList.\FieldSize + 1$
		\EndProcedure
		\Statex
		\Procedure{$\FuncGrow$}{$\VarList$}
		\State $\VarList.\FieldTail.Append(\VarList.\FieldHead)$
		\If{$\VarList.\FieldHeadCapacity = \VarInitCapacity$}
			\State $\VarNewHeadCapacity \gets (\VarGrowthFactor - 1) \times \VarList.\FieldHeadCapacity$
		\Else
			\State $\VarNewHeadCapacity \gets \VarGrowthFactor \times \VarList.\FieldHeadCapacity$
		\EndIf
		\State $\VarList.\FieldHead \gets \FuncNewArray(\VarNewHeadCapacity)$
		\State $\VarList.\FieldCapacity \gets \VarList.\FieldCapacity + \VarNewHeadCapacity$
		\EndProcedure
	\end{algorithmic}
\end{algorithm}

\HdrTimeComplex

I start off again by finding the write cost for $n$ items. Lemma \ref{lem:CapacitySeq} still holds, since growth arrays satisfy the properties used by that proof. In particular, although growth arrays use a different growth algorithm than dynamic arrays, the following claim is still true:

\begin{lemma}
\label{lem:GrowthArraysGrowthFactor}
	The capacity of a growth array grows by the constant factor $\VarGrowthFactor$.
\end{lemma}

\begin{proof}
	I prove that the $\FuncGrow$ algorithm enforces this using induction. I induct on the number of times $\FuncGrow$ is called, $k$, showing that for all natural numbers $k$, $\FuncGrow$ behaves correctly when called the $k$th time. I will let $c_i$ and $c_f$ denote the initial/final capacities and $h_i$ and $h_f$ denote the initial/final head capacities for the $k$th call, respectively.

	For $k = 1$, $c_i = \VarInitCapacity$. I wish to show that $c_f = \VarGrowthFactor\VarInitCapacity$. This happens if and only if the next buffer has size $\Delta c = (\VarGrowthFactor - 1)\VarInitCapacity$, which the algorithm ensures.
	
	For $k > 1$, by induction $c_i = \text{previous }c_f = \VarGrowthFactor^{k - 2}\VarInitCapacity$, and $h_i = \text{previous }h_f = (\VarGrowthFactor^{k - 2} - \VarGrowthFactor^{k - 3})\VarInitCapacity$. I wish to show $c_f = \VarGrowthFactor^{k - 1}\VarInitCapacity$ and $h_f = (\VarGrowthFactor^{k - 1} - \VarGrowthFactor^{k - 2})\VarInitCapacity$. Because $k > 1$, the algorithm will calculate $h_f$ as $\VarGrowthFactor$ times $h_i$. Then
	\begin{align*}
	h_f &= \VarGrowthFactor h_i = \VarGrowthFactor (\VarGrowthFactor^{k - 2} - \VarGrowthFactor^{k - 3})\VarInitCapacity = (\VarGrowthFactor^{k - 1} - \VarGrowthFactor^{k - 2})\VarInitCapacity
	\end{align*}
	and
	\begin{align*}
	c_f &= c_i + h_f = \VarGrowthFactor^{k - 2}\VarInitCapacity + (\VarGrowthFactor^{k - 1} - \VarGrowthFactor^{k - 2})\VarInitCapacity = \VarGrowthFactor^{k - 1}\VarInitCapacity
	\end{align*}
	as desired.
\end{proof}

Since Lemma \ref{lem:GrowthArraysGrowthFactor} has been proven, Lemma \ref{lem:CapacitySeq} and all results based on it must also hold true for growth arrays. Now, I am ready to find the write cost of $\FuncGrow$. Unlike dynamic arrays, $\FuncGrow$ does not make $\VarGrowSeq_i$ writes when the current size is $\VarGrowSeq_i$. In fact, $\FuncGrow$ does not copy \textit{any} items supplied by the user. Writes are only made when a buffer is appended to the tail, since the tail is a dynamic array.

Let $\FuncWritesByGrow(n)$ denote the total number of writes made by $\FuncGrow$, and let $\VarNumItemsTail$ be the size of the tail. Since Corollary \ref{coro:GrowthSeq} also holds true for growth arrays, $\FuncGrow$ is called $|\VarGrowSeq|$ times. A buffer is appended to the tail each time $\FuncGrow$ is called. Thus, the tail's size is
\begin{align*}
\VarNumItemsTail &= |\VarGrowSeq| = |\VarCapacitySeq| - 1 = \VarUseful
\end{align*}
Then the formula for $\FuncWritesByGrow(n)$ is simply $\FuncWritesTail(\VarUseful)$, where $\FuncWritesTail$ denotes the tail's write cost function, that is, the write cost function for dynamic arrays. Finally, adding the writes made by $\FuncAppend$, the formula for $\FuncWrites(n)$ is
\begin{align*}
\FuncWrites(n) = n + \FuncWritesTail(\VarUseful)
\end{align*}
Now, I approximate $\FuncWrites(n)$ using $\sim$. To do this, I will derive the big-O complexity of $\VarUseful$.

\begin{lemma}
\label{lem:VarUsefulIsOLogN}
	$\VarUseful = O(\log n)$.
\end{lemma}

\begin{proof}
	From \ref{lem:CapacitySeq}, $\VarUseful = \max(\ExprMessy, 0)$. As mentioned in Lemma \ref{lem:ToVarUsefulPowerBounds}, for sufficiently large $n$, $\VarUseful = \ExprMessy$. Then
	\begin{align*}
	\VarUseful \sim \ExprMessy \sim \left( \ExprMessyInner \right) \sim \log_{\VarGrowthFactor} n
	\end{align*}
	By [lemma], $O(\VarUseful) = O(\log_{\VarGrowthFactor} n) = O(\log n)$ as desired.
\end{proof}

Now when $\FuncWrites(n)$ is approximated with $\sim$, the $\FuncWritesTail(\VarUseful)$ term disappears:
\begin{align*}
\FuncWrites(n) &= n + \FuncWritesTail(\VarUseful) = n + O(\VarUseful) = n + O(\log n) \sim n
\end{align*}
The last step is true because of Theorem \ref{thm:RemovesLowerOrderTerms}.

\HdrSpaceComplex

Unlike dynamic arrays, growth arrays never throw away buffers. This means that if the current capacity is $c$, then the total length of buffers allocated to store items is also $c$. Typically, however, $\FuncSpace(n) > c$. This is because growth arrays not only store items in buffers, they also store \textbf{references} (or \textbf{pointers}) to the buffers holding the items, in the tail. Thus, the space the tail allocates must also be considered.

We established in [lemma] that $\VarNumItemsTail = \VarUseful$. Then since $\VarCapacitySeq_\VarUseful$ is the space needed to hold items, and $\FuncSpaceTail(\VarUseful)$ is the space the tail needs to hold references, the formula for $\FuncSpace(n)$ is
\begin{align*}
\FuncSpace(n) = \VarCapacitySeq_\VarUseful + \FuncSpaceTail(\VarUseful)
\end{align*}
I now wish to approximate this using $\sim$. First, note that since $\VarCapacitySeq_\VarUseful = \VarGrowthFactor^\VarUseful\VarInitCapacity$, it follows from Lemma \ref{lem:ToVarUsefulPowerBounds} that $n \FluteLeq \VarCapacitySeq_\VarUseful \FluteLess \VarGrowthFactor{n}$. Since $n$ and $\VarGrowthFactor{n}$ are both $O(n)$, [the squeeze theorem] implies that $\VarCapacitySeq_\VarUseful = O(n)$.

Now, consider the ratio $\ExprNToInfty \dfrac {\FuncSpaceTail(\VarUseful)} {\VarCapacitySeq_\VarUseful} = \dfrac {O(\log n)} {O(n)} = O \left( \dfrac{\log n}{n} \right) = 0$. Because it is $0$, from Theorem \ref{thm:RemovesLowerOrderTerms} one can conclude
\begin{align*}
\FuncSpace(n) = \VarCapacitySeq_\VarUseful + \FuncSpaceTail(\VarUseful) \sim \VarCapacitySeq_\VarUseful
\end{align*}
Substituting $\FuncSpace(n)$ for $\VarCapacitySeq_\VarUseful$ in the previous inequality,
\begin{align*}
n \FluteLeq \FuncSpace(n) \FluteLess \VarGrowthFactor{n}
\end{align*}