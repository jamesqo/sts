\subsection{Introduction}

Typically, big-O notation is used to analyze the time or space complexity of a function. In order to highlight the benefit of growth arrays, however, I will use the $\sim$ relation to analyze complexity. This is because growth arrays are only better than dynamic ones by a constant factor for certain operations. For example, dynamic arrays might use roughly $2n$ space for appending $n$ items, while growth arrays would use roughly $n$ space. Even though growth arrays are clearly better in this regard, the big-O space complexity for both data structures would be the same, $O(n)$.

My goal is to be able to compare the coefficients of the highest-order terms in both expressions. For example, I would like to take the ratio $2n : n$, see that it is $2$, and conclude that dynamic arrays allocate roughly twice as much as growth arrays for large $n$. However, big-O notation does not support this.

\subsection{Definitions}

I will mainly use the $\sim$ relation to analyze complexity. It is defined as follows:
\begin{align*}
f \sim g \Biconditional \ExprNToInfty \frac f g = 1
\end{align*}
This is read as ``$f$ is asymptotic to $g$'' or ``$f$ and $g$ are asymptotic.'' {\HdrNote} $f$ and $g$ are used as shorthand to denote $f(n)$ and $g(n)$, respectively.

Notice that while $2n = O(n)$, $2n \not\sim n$. Thus, $\sim$ makes it possible to distinguish between a function that uses $n$ space and one that uses $2n$ space. {\HdrNote} A consequence of this is that bases for logarithms cannot be omitted, like in big-O notation.

I will also introduce two cousins of $\sim$, the relations $\FluteLess$ and $\FluteLeq$. The former is normally defined as $f \prec g \Biconditional \ExprNToInfty \dfrac{f}{g} = 0$. However, for the purpose of this paper their definitions will be:
\begin{align*}
f \FluteLess g &\Biconditional \ExprNToInfty \frac f g < 1\\
f \FluteLeq g &\Biconditional f \FluteLess g \lor f \sim g
\end{align*}
I will call these relations \textbf{asymptotic inequalities}. The above inequalities are read ``$f$ is asymptotically less than $g$'' and ``$f$ is asymptotically less than or equal to $g$,'' respectively.

I will also use little-o notation in this paper, which is defined as follows:
\begin{align*}
f = o(g) \Biconditional \ExprNToInfty \frac{f}{g} = 0
\end{align*}
\subsection{Properties}
\label{subsec:AsymptoticProperties}

I define various properties of the $\sim$ relation here, which proofs in later sections will use. The properties themselves are proved in the appendix.

{\HdrNote} In the following theorems, variables with the letters $f$, $g$, or $h$ denote functions that are positive for sufficiently large $n$.

The following theorem states that $\sim$ is a valid equivalence relation.

\begin{theorem}
\label{thm:EquivalenceRelation}
	$\sim$ is reflexive, transitive, and symmetric.
\end{theorem}

The following theorem states that lower-order terms may be removed: for example, $(n + \log_2 n) \sim n$. This is a property shared with big-O.

\begin{theorem}
\label{thm:RemovesLowerOrderTerms}
	$f + o(f) \sim f$.
\end{theorem}

\begin{corollary}
\label{coro:PlusConstant}
	If $f$ is unbounded, then $f + c \sim f$ for any constant $c$.
\end{corollary}

The following theorem states that $\sim$ respects addition, multiplication, and division. This is also a property shared with big-O.

\begin{theorem}
\label{thm:MergesOverOps}
	If $f \sim \Tweak{f}$ and $g \sim \Tweak{g}$, then
	\begin{align*}
	f + g &\sim \Tweak{f} + \Tweak{g}\\
	fg &\sim \Tweak{f}\Tweak{g}\\
	\frac{f}{g} &\sim \frac {\Tweak{f}} {\Tweak{g}}
	\end{align*}
\end{theorem}

\begin{corollary}
\label{coro:BothSides}
	A $\sim$ relation is preserved when both sides are added, multiplied, or divided by the same positive quantity.
\end{corollary}

The following theorem states that asymptotic functions belong to the same big-O class.

\begin{theorem}
\label{thm:SameBigOClass}
	If $f \sim g$ and $g = O(h)$, then $f = O(h)$.
\end{theorem}

The following theorem states that asymptotic functions can be interchanged in an asymptotic inequality.

\begin{theorem}
\label{thm:InterchangeableInInequality}
	If $f \sim \Tweak{f}$ and $\Tweak{f} \FluteLess \TweakSecond{f}$, then $f \FluteLess \TweakSecond{f}$. If $g \sim \TweakSecond{g}$ and $\Tweak{g} \FluteLess \TweakSecond{g}$, then $\Tweak{g} \FluteLess g$.\\
	These statements also hold for $\FluteLeq$.
\end{theorem}

The following theorem states that asymptotic inequalities can be algebraically manipulated, in the same vein as Corollary \ref{coro:BothSides}.

\begin{theorem}
\label{thm:BothSidesInequality}
	A $\FluteLess$ inequality is preserved when both sides are multiplied or divided by the same positive quantity.\\
	This statement also holds for $\FluteLeq$.
\end{theorem}

\begin{corollary}
\label{coro:BothSidesInequalityInvert}
	If $f \FluteLess g$, then $\dfrac{1}{g} \FluteLess \dfrac{1}{f}$.\\
	This statement also holds for $\FluteLeq$.
\end{corollary}

Notice that the above theorem does not include addition. Asymptotic inequalities do not always hold if the same quantity is added to both sides; for example, $1 \FluteLess 2$, but $n + 1 \sim n + 2$. In order for them to hold, certain conditions must be met.

\begin{theorem}
\label{thm:BothSidesInequalityAdd}
	Suppose $f \FluteLess g$. If $\ExprNToInfty \dfrac{f}{h} > 0$ and $g = O(h)$, then $f + h \FluteLess g + h$.\\
	This statement also holds for $\FluteLeq$.
\end{theorem}