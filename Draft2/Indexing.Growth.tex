\HdrGrowthArrayImpl

Indexing a growth array is slower and more complex than indexing a dynamic array. However, this does not mean that accessing data of a growth array is necessarily slower than accessing that of a dynamic array. Refer to Sections \ref{sec:Iterating} and \ref{sec:CopyArray} for methods to quickly access growth arrays' data.

Despite growth arrays' relatively poor indexing performance, I include this section for two reasons. 1) Dynamic arrays are indexed frequently. In order for growth arrays to be a viable replacement for them, it should be possible to index growth arrays too. 2) I wish to show that it is possible to index growth arrays in constant time.

\HdrLogarithmicImpl

The first algorithm I will demonstrate is a na\"{i}ve implementation that runs in logarithmic time:

\begin{algorithm}[H]
	\begin{algorithmic}
		\Function{$\FuncGetItem$}{$\VarList,\ \ParamIndex$}
			\State $\LclIndex \gets \ParamIndex$
			\For{$\LclBuffer \TextIn \VarList.\FieldTail$}
				\If{$\LclIndex < \LclBuffer.\FieldLength$}
					\State \Return $\LclBuffer[\LclIndex]$
				\EndIf
				\State $\LclIndex \gets \LclIndex - \LclBuffer.\FieldLength$
			\EndFor
			\State \Return $\VarList.\FieldHead[\LclIndex]$
		\EndFunction
	\end{algorithmic}
\end{algorithm}

I know that this algorithm runs in $O(\log n)$ time since it was shown earlier that the size of the tail, $\VarNumPointersTail$, equals $\VarUseful$, and (from Lemma \ref{lem:VarUsefulIsOLogN}) that $\VarUseful = O(\log n)$. Aside from iteration over the tail, all statements run in constant time, so the time complexity is precisely $O(\log n)$.

I present this algorithm alongside the constant time one because it is easier to understand, and the latter is often slower if special circumstances are not met.

\HdrConstantImpl

In this section, I will present a constant time algorithm for indexing a growth array. Before I do so, however, I must establish a mathematical justification for it.

Suppose I want to get the $i$th element of a growth array, where $i$ is zero-based. I assume that $i$ is a valid index, or that $0 \leq i < \VarList.\FieldSize$. In order to locate the desired element, I must find two things: the buffer that holds the item, and the index of the item within that buffer.

Now, consider that every buffer except the head is stored inside the tail. Thus, such buffers can be uniquely identified by their index in the tail. I will refer to this quantity as the \textbf{buffer index} and denote it $\VarIndexBuffer$. I wish to assign the head a buffer index as well, so that each buffer has a unique ID. Since the head follows the tail's last buffer, which has $\VarIndexBuffer = \VarNumPointersTail - 1$, I will let the head's $\VarIndexBuffer$ be $\VarNumPointersTail$.

I define a helper function, $\FuncGetBuffer$, that gets the buffer associated with a given $\VarIndexBuffer$. $\VarIndexBuffer$ is assumed to be valid; that is, $0 \leq \VarIndexBuffer \leq \VarNumPointersTail$.

\begin{algorithm}[H]
	\begin{algorithmic}
		\Function{$\FuncGetBuffer$}{$\VarList,\ \LclIndexBuffer$}
			\If{$\LclIndexBuffer < \VarList.\FieldTail.\FieldSize$}
				\State \Return $\VarList.\FieldTail[\LclIndexBuffer]$
			\Else
				\State \Return $\VarList.\FieldHead$
			\EndIf
		\EndFunction
	\end{algorithmic}
\end{algorithm}

I will call the index of the desired element within the buffer the \textbf{element index}, and denote it $\VarIndexElement$.

If I define a helper function $\FuncDecompose$ that returns the $\VarIndexBuffer$ and $\VarIndexElement$ associated with $i$ in an ordered pair, then $\FuncGetItem$ may be written as follows:

\begin{algorithm}[H]
	\begin{algorithmic}
		\Function{$\FuncGetItem$}{$\VarList,\ \ParamIndex$}
			\State $(\LclIndexBuffer,\ \LclIndexElement) \gets \FuncDecompose(\ParamIndex)$
			\State \Return $\VarList.\FuncGetBuffer(\LclIndexBuffer)[\LclIndexElement]$
		\EndFunction
	\end{algorithmic}
\end{algorithm}

Now, the task is to find formulae for $\VarIndexBuffer$ and $\VarIndexElement$ in terms of $i$, in order to implement $\FuncDecompose$.

% Todo: There's an awful lot of lemmas. Make some into theorems.
\begin{lemma}
	The formulae for $\VarIndexBuffer$ and $\VarIndexElement$ are
	\begin{align*}
	\VarIndexBuffer &= \VarUseful|_{n = i + 1}\\
	\VarIndexElement &= i - \VarGrowSeq_{\VarIndexBuffer - 1}
	\end{align*}
	(When $\VarIndexBuffer = 0$, $\VarGrowSeq_{-1} = 0$ by convention.)
\end{lemma}

% Todo: "As shown earlier"
\begin{proof}
	Appending new items does not change the index of an item that is already in the list. Thus, finding the element at index $i$ is equivalent to finding the last element when $n = i + 1$.
	
	The last element always resides in the head, so $\VarIndexBuffer = \VarNumPointersTail|_{n = i + 1}$. As shown earlier, $\VarNumPointersTail = \VarUseful$, so $\VarIndexBuffer = \VarUseful|_{n = i + 1}$.
	
	To determine $\VarIndexElement$, consider the identity:
	\begin{align*}
	n = \text{\# of items in tail} + \text{\# of items in head}
	\end{align*}
	First, I will determine the number of items in the tail. In the case where $\VarIndexBuffer = 0$, then because $\VarNumPointersTail = \VarIndexBuffer$, the tail contains $0$ buffers and thus $0$ items (which equals $\VarGrowSeq_{-1}$). If $\VarIndexBuffer > 0$, the number of items in the tail is the last size at which $\FuncGrow$ was called, or the last term of $\VarGrowSeq$. This quantity is $\VarGrowSeq_{\VarUseful - 1}$, or $\VarGrowSeq_{\VarIndexBuffer - 1}$.
	
	The desired item is the growth array's last element, which implies it is also the head's last element. Thus $\VarIndexElement$ is the head's last valid index, so the head's size is $\VarIndexElement + 1$. Finally, from the premise, $n = i + 1$. Substituting all values into the above identity, I receive
	\begin{align*}
	i + 1 &= \VarGrowSeq_{\VarIndexBuffer - 1} + (\VarIndexElement + 1)\\
	\VarIndexElement &= i - \VarGrowSeq_{\VarIndexBuffer - 1}
	\end{align*}
	completing the proof.
\end{proof}

Using the formulae for $\VarUseful$ and $\VarGrowSeq_i$, I now implement the $\FuncDecompose$ function:

\begin{algorithm}[H]
	\begin{algorithmic}
		\Function{$\FuncDecompose$}{$\ParamIndex$}
			\State $\LclIndexBuffer \gets \max(\left\lceil \log_{\VarGrowthFactor} (\ParamIndex + 1) - \log_{\VarGrowthFactor} \VarInitCapacity \right\rceil, 0)$
			\If{$\VarIndexBuffer > 0$}
				\State $\LclIndexElement \gets \ParamIndex - \VarGrowthFactor^{\LclIndexBuffer - 1} \times \VarInitCapacity$
			\Else
				\State $\VarIndexElement \gets \ParamIndex$
			\EndIf
			\State \Return $(\LclIndexBuffer,\ \LclIndexElement)$
		\EndFunction
	\end{algorithmic}
\end{algorithm}

Clearly, $\FuncDecompose$ runs in constant time. Despite that, it appears to be quite expensive: normally, logarithms and exponentiation require use of floating-point instructions, which are slower than integer-based instructions. However, in the special case where $\VarGrowthFactor = 2$ and $\VarInitCapacity = 2^\VarLogInitCapacity$ for some constant whole number $\VarLogInitCapacity$, I claim that $\VarIndexBuffer$ and $\VarIndexElement$ can be computed without use of floating-point instructions.

To see this, first note that $\VarIndexBuffer$ becomes $\max(\left\lceil \log_2 (i + 1) \right\rceil - \VarLogInitCapacity, 0)$. There is a constant time algorithm to compute $\left\lceil \log_2 k \right\rceil$ without using floating-point for any $k \in \mathbb{N}$, which I will not discuss because it involves concepts outside the scope of this paper. (An implementation of the algorithm, however, may be found in the links at Section \ref{sec:Implementations}.) Equipped with such an algorithm, it is easy to see that the whole expression can be computed without use of floating-point.

In calculating $\VarIndexElement$, the only potential use of floating-point instructions comes from $\VarGrowthFactor^{\VarIndexBuffer - 1}$. When $\VarGrowthFactor = 2$, however, this can be calculated with a simple bit shift. Thus, it is not necessary to use floating-point instructions to calculate either $\VarIndexBuffer$ or $\VarIndexElement$.