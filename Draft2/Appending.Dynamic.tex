\HdrDynArrayImpl

I will implement appending for dynamic arrays first. Let $\VarList$ be a dynamic array. The following constants are used by the algorithm:

\begin{algorithm}[H]
	\caption{Appending \TextDynamicArray}
	\begin{algorithmic}[1]
		\Procedure{$\FuncAppend$}{$\VarList,\ \ParamItem$}
			\If{$\VarList.\FieldFull$}
				\State $\VarList.\FuncGrow()$
			\EndIf
			
			\State $\VarList.\FieldBuffer[\VarList.\FieldSize] \gets \ParamItem$
			\State $\VarList.\FieldSize \gets \VarList.\FieldSize + 1$
		\EndProcedure
		\Statex
		\Procedure{$\FuncGrow$}{$\VarList$}
			\State $\LclNewBuffer \gets \FuncNewArray(\VarGrowthFactor \times \VarList.\FieldSize)$
			\State $\FuncArrayCopy(\VarList.\FieldBuffer,\ \LclNewBuffer,\ \VarList.\FieldSize)$
			\State $\VarList.\FieldBuffer \gets \LclNewBuffer$
		\EndProcedure
	\end{algorithmic}
\end{algorithm}

\HdrTimeComplex

It is well-known that appending one item to a dynamic array takes amortized $O(1)$ time. Thus, appending $n$ items takes $O(n)$ time. It will be shown later, however, that growth arrays also take $O(n)$ time for this task, so this does not let us compare time complexities.

I wish to find the ratio $\frac{\FuncTimeDynamicArray(n)}{\FuncTimeGrowthArray(n)}$, where $\FuncTimeDynamicArray(n)$ and $\FuncTimeGrowthArray(n)$ denote the average time needed to append $n$ items for dynamic and growth arrays, respectively. This represents how much faster growth arrays are at appending than dynamic ones. I will approximate this ratio using the following question: Suppose $n$ elements are appended to an empty collection. How many times is an element stored in an array? I will term the answer to this question the \textbf{write cost} of $n$ appends, and denote it $\FuncWrites(n)$.

I will assume that $\FuncTime(n) \propto \FuncWrites(n)$, or that the average time for $n$ appends is proportional to their write cost (although this is an oversimplification). Then $\frac{\FuncTimeDynamicArray(n)}{\FuncTimeGrowthArray(n)} = \frac{\FuncWritesDynamicArray(n)}{\FuncWritesGrowthArray(n)}$, where $\FuncWritesDynamicArray$ and $\FuncWritesGrowthArray$ are respectively the write cost functions for dynamic and growth arrays. In this section, I will derive the formula for $\FuncWritesDynamicArray(n)$. Because this section concerns dynamic arrays only, I will denote it $\FuncWrites(n)$.

In the code for $\FuncAppend$, one array store is performed unconditionally, so it is apparent that $\FuncWrites(n) \geq n$ after $n$ appends. However, $\FuncGrow$ also does some writing via $\FuncArrayCopy$, so in order to find a precise formula for $\FuncWrites(n)$, I need to analyze when $\FuncGrow$ is called. To do this, I use the following lemma:

\begin{lemma}
\label{lem:CapacitySeq}
	Let $\VarList$ be a dynamic array. Let its \textbf{capacity sequence}, $\VarCapacitySeq$, be the range of values for $\VarList.\FieldCapacity$ as $n$ items are appended. For $n = 0$, trivially $\VarCapacitySeq = (\VarInitCapacity)$. For $n > 0$,
	\begin{align*}
	\VarCapacitySeq = \VarInitCapacity,\ \VarGrowthFactor\VarInitCapacity,\ \VarGrowthFactor^2\VarInitCapacity,\ \ldots\ \VarGrowthFactor^{\max(\ExprMessy, 0)}\VarInitCapacity
	\end{align*}
\end{lemma}

\begin{proof}
	I use the following properties of dynamic arrays:
	\begin{enumerate}
		\item The capacity of an empty dynamic array is $\VarInitCapacity$.
		\item The capacity of a dynamic array can only grow by $\VarGrowthFactor$.
		\item The capacity is as small as possible. Put formally, if $\VarCapacitySeq_i$ is the capacity for $n$ items, then $\VarCapacitySeq_i \geq n$ but $n > \VarCapacitySeq_{i - 1}$. (By convention, $\VarCapacitySeq_{-1} = 0$.)
	\end{enumerate}
	Assumption (1) immediately shows $\VarCapacitySeq_0 = \VarInitCapacity$. Assumption (2) shows that if $\VarGrowthFactor^i\VarInitCapacity$ is the current capacity, then $\VarGrowthFactor^{i + 1}\VarInitCapacity$ must be the next capacity. By induction, $\VarCapacitySeq = \left( \VarGrowthFactor^i\VarInitCapacity \right)_{i = 0}^\VarUseful$ for some whole number $\VarUseful$.
	
	The final value of the sequence, $\VarCapacitySeq_\VarUseful$, is the capacity needed to store $n$ items. By assumption (3), $\VarCapacitySeq_\VarUseful \geq n > \VarCapacitySeq_{\VarUseful - 1}$. Consider the case when $n > \VarInitCapacity$: it must be true that $\VarCapacitySeq_\VarUseful > \VarInitCapacity$, so $\VarUseful \geq 1$. Since $\VarUseful - 1 \geq 0$, $\VarCapacitySeq_\VarUseful = \VarGrowthFactor^\VarUseful\VarInitCapacity$ and $\VarCapacitySeq_{\VarUseful - 1} = \VarGrowthFactor^{\VarUseful - 1}\VarInitCapacity$. Then
	\begin{align*}
	\VarGrowthFactor^\VarUseful\VarInitCapacity &\geq n > \VarGrowthFactor^{\VarUseful - 1}\VarInitCapacity\\
	\VarGrowthFactor^\VarUseful &\geq \frac{n}{\VarInitCapacity} > \VarGrowthFactor^{\VarUseful - 1}\\
	\VarUseful &\geq \log_{\VarGrowthFactor} n - \log_{\VarGrowthFactor} \VarInitCapacity > \VarUseful - 1\\
	\end{align*}
	Since $\VarUseful$ is an integer,
	\begin{align*}
	\VarUseful &= \ExprMessy
	\end{align*}
	Now, consider the case when $n \leq \VarInitCapacity$. By assumption (3), $n > \VarCapacitySeq_{\VarUseful - 1}$. $\VarUseful - 1$ must then equal $-1$, since any other value would imply $n > \VarCapacitySeq_{\VarUseful - 1} \geq \VarInitCapacity$. Thus $\VarUseful = 0$.
	
	It was shown that $\VarUseful \geq 1 \geq 0$ for the first case, and it can be shown that $\ExprMessy \leq 0$ for the second case. Therefore, a general formula for $\VarUseful$ is as follows:
	\begin{align*}
	\VarUseful &= \max(\ExprMessy, 0)
	\end{align*}
	The final term in the sequence is $\VarGrowthFactor^\VarUseful\VarInitCapacity = \VarGrowthFactor^{\max(\ExprMessy, 0)}\VarInitCapacity$, completing the proof.
\end{proof}

\begin{corollary}
\label{coro:GrowthSeq}
	Let the \textbf{growth sequence}, $\VarGrowSeq$, of $\VarList$ be the sizes at which $\FuncGrow$ is called when $n$ items are appended to $\VarList$. Then $\VarGrowSeq = \VarCapacitySeq \setminus \left\{ \VarCapacitySeq_\VarUseful \right\}$.
\end{corollary}

\begin{proof}
	If $\VarCapacitySeq_i$ exists and $i \geq 1$, then clearly $\FuncGrow$ must have been called when the size was $\VarCapacitySeq_{i - 1}$, so $\VarCapacitySeq_{i - 1} \in \VarGrowSeq$. Then $\VarGrowSeq$ contains every term in $\VarCapacitySeq$ except for the last, $\VarCapacitySeq_\VarUseful$, as the corollary states.
\end{proof}

When $\FuncGrow$ is called and the current size is $\VarGrowSeq_i$, the algorithm copies $\VarGrowSeq_i$ items. Then the total number of items copied when $n$ items are appended is:
\begin{align*}
\sum_i {\VarGrowSeq_i} &= \VarInitCapacity + \VarGrowthFactor\VarInitCapacity + \ldots + \VarGrowthFactor^{\VarUseful - 1}\VarInitCapacity\\
&= \left( \frac{\VarGrowthFactor^{\VarUseful} - 1}{\VarGrowthFactor - 1} \right) \VarInitCapacity
\end{align*}
Counting the writes made per item by $\FuncAppend$, an explicit formula for $\FuncWrites(n)$ is as follows:
\begin{align*}
\FuncWrites(n) = n + \left( \frac{\VarGrowthFactor^\VarUseful - 1}{\VarGrowthFactor - 1} \right) \VarInitCapacity
\end{align*}
Now, my goal is to approximate $\FuncWrites(n)$ with $\sim$. To make this easier to do, I will asymptotically bound $\VarGrowthFactor^\VarUseful$ with respect to $n$.

\begin{lemma}
\label{lem:ToVarUsefulPowerBounds}
	$\frac n \VarInitCapacity \FluteLeq \VarGrowthFactor^\VarUseful \FluteLess \frac {\VarGrowthFactor{n}} \VarInitCapacity$. That is, $\frac n \VarInitCapacity \leq \VarGrowthFactor^\VarUseful < \frac {\VarGrowthFactor{n}} \VarInitCapacity$ for sufficiently large $n$.
\end{lemma}

\begin{proof}
	It was shown in Lemma \ref{lem:CapacitySeq} that if $n > \VarInitCapacity$, $\VarUseful = \ExprMessy \geq 1$.  Now, note that $\VarUseful$ may be written as $\ExprMessyAlt$ for such $n$. Then,
	\begin{align*}
	\ExprMessyAltInner &\leq \VarUseful < \ExprMessyAltInner + 1\\
	\frac{n}{\VarInitCapacity} &\leq \VarGrowthFactor^\VarUseful < \frac{\VarGrowthFactor{n}}{\VarInitCapacity}
	\end{align*}
	as desired.
\end{proof}

I can build on this inequality to receive asymptotic bounds for $\FuncWrites(n)$:
\begin{align*}
\frac n \VarInitCapacity &\FluteLeq \VarGrowthFactor^\VarUseful \FluteLess \frac {\VarGrowthFactor{n}} \VarInitCapacity\\
\frac n \VarInitCapacity &\FluteLeq \VarGrowthFactor^\VarUseful - 1 \FluteLess \frac {\VarGrowthFactor{n}} \VarInitCapacity &&\text{(Corollary \ref{coro:PlusConstant}, Theorem \ref{thm:InterchangeableInInequality})}\\
\frac n {\VarGrowthFactor - 1} &\FluteLeq \left( \frac{\VarGrowthFactor^\VarUseful - 1}{\VarGrowthFactor - 1} \right) \VarInitCapacity \FluteLess \frac {\VarGrowthFactor{n}} {\VarGrowthFactor - 1} &&\text{(Theorem \ref{thm:BothSidesInequality})}\\
\left( \frac{\VarGrowthFactor}{\VarGrowthFactor - 1} \right) n &\FluteLeq \FuncWrites(n) \FluteLess \left( \frac{2\VarGrowthFactor - 1}{\VarGrowthFactor - 1} \right) n &&\text{(Theorem \ref{thm:BothSidesInequalityAdd})}
\end{align*}
This means that, for large $n$, the number of items written to append $n$ items is roughly between the two bounds shown. For example, suppose $\VarInitCapacity = 4$, $\VarGrowthFactor = 2$, and $32$ items are appended. $\FuncGrow$ is called at the sizes $4$, $8$, and $16$, so $4 + 8 + 16 = 28$ items are copied. Counting the $32$ writes made by $\FuncAppend$, one per item, the write cost is $\FuncWrites(32) = 28 + 32 = 60$. This is approximately equal to the lower bound, $\frac{2}{2 - 1} \cdot 32 = 64$.

Now, suppose that one more item is added so that $n = 33$. $\FuncGrow$ is called again, so the number of items copied becomes $4 + 8 + 16 + 32 = 60$. The write cost becomes $\FuncWrites(33) = 33 + 60 = 93$. This is approximately equal to the upper bound, $\frac{2 \cdot 2 - 1}{2 - 1} \cdot 33 = 99$.

Intuitively, it makes sense that $\FuncWrites(n)$ should jump from the lower bound to the upper bound when $n = \VarGrowSeq_i + 1$, since at that point $O(n)$ many items are copied from the old to the new buffer. (In the previous example, $\VarGrowSeq_i + 1$ is $17$.) As new items are added, however, both $n$ and $\FuncWrites(n)$ only increase by $1$, and the ratio $\frac {\FuncWrites(n)} n$ slowly dwindles toward the lower bound until $n$ reaches $\VarGrowSeq_{i + 1} + 1$.

\HdrSpaceComplex

I wish to find the space allocated when $n$ items are appended to a collection. I call this quantity the \textbf{space cost}, denote it $\FuncSpace(n)$, and define it as the total length of buffers allocated by $n$ appends. Now, I derive a formula for $\FuncSpace(n)$.

First, from the definition of $\VarList.\FieldCapacity$, note that a dynamic array's capacity is the length of the buffer it stores its items in. Then a buffer of length $c$ is allocated at some point if and only if $c \in \VarCapacitySeq$. Then the total length of those buffers is
\begin{align*}
\FuncSpace(n) &= \sum_i {\VarCapacitySeq_i}\\
&= \VarInitCapacity + \VarGrowthFactor\VarInitCapacity + \VarGrowthFactor^2\VarInitCapacity + \ldots + \VarGrowthFactor^\VarUseful\VarInitCapacity\\
&= \left( \frac{\VarGrowthFactor^{\VarUseful + 1} - 1}{\VarGrowthFactor - 1} \right) \VarInitCapacity
\end{align*}
Using Lemma \ref{lem:ToVarUsefulPowerBounds} again, I asymptotically bound $\FuncSpace(n)$:
\begin{align*}
\frac n \VarInitCapacity &\FluteLeq \VarGrowthFactor^\VarUseful \FluteLess \frac {\VarGrowthFactor{n}} \VarInitCapacity\\
\frac {\VarGrowthFactor n} \VarInitCapacity &\FluteLeq \VarGrowthFactor^{\VarUseful + 1} \FluteLess \frac {\VarGrowthFactor^2 n} \VarInitCapacity &&\text{(Theorem \ref{thm:BothSidesInequality})}\\
\frac {\VarGrowthFactor n} \VarInitCapacity &\FluteLeq \VarGrowthFactor^{\VarUseful + 1} - 1 \FluteLess \frac {\VarGrowthFactor^2 n} \VarInitCapacity &&\text{(Corollary \ref{coro:PlusConstant}, Theorem \ref{thm:InterchangeableInInequality})}\\
\left( \frac{\VarGrowthFactor}{\VarGrowthFactor - 1} \right) n &\FluteLeq \FuncSpace(n) \FluteLess \left( \frac{\VarGrowthFactor^2}{\VarGrowthFactor - 1} \right) n &&\text{(Theorem \ref{thm:BothSidesInequality})}
\end{align*}
Using the same intuition from the time complexity section, $\FuncSpace(n)$ approaches the lower bound as $n \to \VarGrowSeq_i$, and jumps to the upper bound when $n = \VarGrowSeq_i + 1$. If $100$ items were appended to a dynamic array with $\VarInitCapacity = 11$ and $\VarGrowthFactor = 3$, for example, the wasted space $s(100) - 100$ would be approximately $\frac{3^2}{3 - 1} \cdot 100 - 100 = 350$. If $n$ were $99$, however, it would be approximately $\frac{3}{3 - 1} \cdot 99 - 99 = 49.5$.