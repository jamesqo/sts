\HdrDynArrayImpl

I will implement appending for dynamic arrays first. Let $\VarList$ be a dynamic array. The following constants are used by the algorithm:

\begin{algorithm}
	\begin{algorithmic}[1]
		\Procedure{$\FuncAppend$}{$\VarList,\ \ParamItem$}
			\If{$\VarList.\FieldFull$}
				\State $\VarList.\FuncGrow()$
			\EndIf
			
			\State $\VarList.\FieldBuffer[\VarList.\FieldSize] \gets \ParamItem$
			\State $\VarList.\FieldSize \gets \VarList.\FieldSize + 1$
		\EndProcedure
		\Statex
		\Procedure{$\FuncGrow$}{$\VarList$}
			\State $\LclNewBuffer \gets \FuncNewArray(\VarGrowthFactor \times \VarList.\FieldSize)$
			\State $\FuncArrayCopy(\VarList.\FieldBuffer,\ \LclNewBuffer,\ \VarList.\FieldSize)$
			\State $\VarList.\FieldBuffer \gets \LclNewBuffer$
		\EndProcedure
	\end{algorithmic}
\end{algorithm}

\HdrTimeComplex

% Todo: CITE
It is well-known that appending one item to a dynamic array takes amortized $O(1)$ time. Thus, appending $n$ items takes $O(n)$ time. However, this fact is not very useful for the purpose of comparison--- later, it will be shown that growth arrays also take $O(n)$ time to append $n$ items.

It would be helpful if the ratio $\dfrac{\FuncTimeDynamicArray(n)}{\FuncTimeGrowthArray(n)}$ could be determined, where $\FuncTimeDynamicArray(n)$ and $\FuncTimeGrowthArray(n)$ denote the average time needed to append $n$ items for dynamic and growth arrays, respectively. This represents how much faster growth arrays are than dynamic ones. However, there is no way to quantify these functions without empirical measurements. Thus, I will approximate this ratio through a different technique: Suppose $n$ elements are appended to an empty collection. How many times is an element stored in an array? I will term the answer to this question the \textbf{write cost} of $n$ appends, and denote it $\FuncWrites(n)$.

I will na\"{i}vely assume that $\FuncTime(n) \propto \FuncWrites(n)$, or that the average time for $n$ appends is proportional to their write cost. Then $\dfrac{\FuncTimeDynamicArray(n)}{\FuncTimeGrowthArray(n)} = \dfrac{\FuncWritesDynamicArray(n)}{\FuncWritesGrowthArray(n)}$, where $\FuncWritesDynamicArray$ and $\FuncWritesGrowthArray$ are respectively the write cost functions for dynamic and growth arrays. In this section, I will derive the formula for $\FuncWritesDynamicArray(n)$. Because this section concerns dynamic arrays only, I will write it as $\FuncWrites(n)$.

In the code for $\FuncAppend$, one array store is performed unconditionally, so it is apparent that $\FuncWrites(n) \geq n$ after $n$ appends. However, $\FuncGrow$ also does some writing, so in order to find a precise formula for $\FuncWrites(n)$, I need to analyze when $\FuncGrow$ is called. To do this, I use the following lemma:

\begin{lemma}
\label{lem:CapacitySeq}
	Let $\VarList$ be a dynamic array. Let its \textbf{capacity sequence}, $\VarCapacitySeq$, be the range of values for $\VarList.\FieldCapacity$ as $n$ items are appended. For $n = 0$, trivially $\VarCapacitySeq = (\VarInitCapacity)$. For $n > 0$,
	\begin{align*}
	\VarCapacitySeq = \VarInitCapacity,\ \VarGrowthFactor\VarInitCapacity,\ \VarGrowthFactor^2\VarInitCapacity,\ \ldots\ \VarGrowthFactor^{\max(\ExprMessy, 0)}\VarInitCapacity
	\end{align*}
\end{lemma}

\begin{proof}
	I use the following properties of dynamic arrays:
	\begin{enumerate}
		\item The capacity of an empty dynamic array is $\VarInitCapacity$.
		\item The capacity of a dynamic array can only grow by $\VarGrowthFactor$.
		\item The capacity is as small as possible. Put formally, if $\VarCapacitySeq_i$ is the capacity for $n$ items, then $\VarCapacitySeq_i \geq n$ but $n > \VarCapacitySeq_{i - 1}$. (By convention, $\VarCapacitySeq_{-1} = 0$.)
	\end{enumerate}
	Assumption (1) immediately shows $\VarCapacitySeq_0 = \VarInitCapacity$. Assumption (2) shows that if $\VarGrowthFactor^i\VarInitCapacity$ is the current capacity, then $\VarGrowthFactor^{i + 1}\VarInitCapacity$ must be the next capacity. By induction, $\VarCapacitySeq = \left( \VarGrowthFactor^i\VarInitCapacity \right)_{i = 0}^\VarUseful$ for some whole number $\VarUseful$.
	
	The final value of the sequence, $\VarCapacitySeq_\VarUseful$, is the capacity needed to store $n$ items. By assumption (3), $\VarCapacitySeq_\VarUseful \geq n > \VarCapacitySeq_{\VarUseful - 1}$. Consider the case when $n > \VarInitCapacity$: it must be true that $\VarCapacitySeq_\VarUseful > \VarInitCapacity$, so $\VarUseful \geq 1$. Since $\VarUseful - 1 \geq 0$, $\VarCapacitySeq_\VarUseful = \VarGrowthFactor^\VarUseful\VarInitCapacity$ and $\VarCapacitySeq_{\VarUseful - 1} = \VarGrowthFactor^{\VarUseful - 1}\VarInitCapacity$. Then
	\begin{align*}
	\VarGrowthFactor^\VarUseful\VarInitCapacity &\geq n > \VarGrowthFactor^{\VarUseful - 1}\VarInitCapacity\\
	\VarGrowthFactor^\VarUseful &\geq \frac{n}{\VarInitCapacity} > \VarGrowthFactor^{\VarUseful - 1}\\
	\VarUseful &\geq \log_{\VarGrowthFactor} n - \log_{\VarGrowthFactor} \VarInitCapacity > \VarUseful - 1\\
	\end{align*}
	Since $\VarUseful$ is an integer,
	\begin{align*}
	\VarUseful &= \ExprMessy
	\end{align*}
	Now, consider the case when $n \leq \VarInitCapacity$. By assumption (3), $n > \VarCapacitySeq_{\VarUseful - 1}$. $\VarUseful - 1$ must then equal $-1$, since any other value would imply $n > \VarCapacitySeq_{\VarUseful - 1} \geq \VarInitCapacity$. Thus $\VarUseful = 0$.
	
	It was shown that $\VarUseful \geq 1 \geq 0$ for the first case, and it can be shown that $\ExprMessy \leq 0$ for the second case. Therefore, a general formula for $\VarUseful$ is as follows:
	\begin{align*}
	\VarUseful &= \max(\ExprMessy, 0)
	\end{align*}
	The final term in the sequence is $\VarGrowthFactor^\VarUseful\VarInitCapacity = \VarGrowthFactor^{\max(\ExprMessy, 0)}\VarInitCapacity$, completing the proof.
\end{proof}

\begin{corollary}
\label{coro:GrowthSeq}
	Let the \textbf{growth sequence}, $\VarGrowSeq$, of $\VarList$ be the sizes at which $\FuncGrow$ is called when $n$ items are appended to $\VarList$. Then $\VarGrowSeq = \VarCapacitySeq \setminus \left\{ \VarCapacitySeq_\VarUseful \right\}$.
\end{corollary}

\begin{proof}
	% Todo: Better language here? 'exists' is confusing
	If $\VarCapacitySeq_i$ exists and $i \geq 1$, then clearly $\FuncGrow$ must have been called when the size was $\VarCapacitySeq_{i - 1}$, so $\VarCapacitySeq_{i - 1} \in \VarGrowSeq$. Then $\VarGrowSeq$ contains every term in $\VarCapacitySeq$ except for the last, $\VarCapacitySeq_\VarUseful$, as the corollary states.
\end{proof}

When $\FuncGrow$ is called and the current size is $\VarGrowSeq_i$, the algorithm copies $\VarGrowSeq_i$ items. Then the total number of items copied when $n$ items are appended is:
\begin{align*}
\sum_i {\VarGrowSeq_i} &= \VarInitCapacity + \VarGrowthFactor\VarInitCapacity + \ldots + \VarGrowthFactor^{\VarUseful - 1}\VarInitCapacity\\
&= \left( \frac{\VarGrowthFactor^{\VarUseful} - 1}{\VarGrowthFactor - 1} \right) \VarInitCapacity
\end{align*}
Counting the writes made per item by $\FuncAppend$, an explicit formula for $\FuncWrites(n)$ is as follows:
\begin{align*}
\FuncWrites(n) = n + \left( \frac{\VarGrowthFactor^\VarUseful - 1}{\VarGrowthFactor - 1} \right) \VarInitCapacity
\end{align*}
Now, my goal is to approximate $\FuncWrites(n)$ with $\sim$. To make this easier to do, I will asymptotically bound $\VarGrowthFactor^\VarUseful$ with respect to $n$.

\begin{lemma}
\label{lem:ToVarUsefulPowerBounds}
	$\dfrac n \VarInitCapacity \FluteLeq \VarGrowthFactor^\VarUseful \FluteLess \dfrac {\VarGrowthFactor{n}} \VarInitCapacity$. That is, $\dfrac n \VarInitCapacity \leq \VarGrowthFactor^\VarUseful < \dfrac {\VarGrowthFactor{n}} \VarInitCapacity$ for sufficiently large $n$.
\end{lemma}

\begin{proof}
	It was shown in Lemma \ref{lem:CapacitySeq} that if $n > \VarInitCapacity$, $\VarUseful = \ExprMessy \geq 1$.  Now, note that $\VarUseful$ may be written as $\ExprMessyAlt$ for such $n$. Then,
	\begin{align*}
	\ExprMessyAltInner \leq \VarUseful &< \ExprMessyAltInner + 1\\
	\frac{n}{\VarInitCapacity} \leq \VarUseful &< \frac{\VarGrowthFactor{n}}{\VarInitCapacity}
	\end{align*}
	as desired.
\end{proof}

I can build on this inequality to receive asymptotic bounds for $\FuncWrites(n)$:
\begin{align*}
\frac n \VarInitCapacity &\FluteLeq \VarGrowthFactor^\VarUseful \FluteLess \frac {\VarGrowthFactor{n}} \VarInitCapacity\\
\frac n \VarInitCapacity &\FluteLeq \VarGrowthFactor^\VarUseful - 1 \FluteLess \frac {\VarGrowthFactor{n}} \VarInitCapacity\\
\frac n {\VarGrowthFactor - 1} &\FluteLeq \left( \frac{\VarGrowthFactor^\VarUseful - 1}{\VarGrowthFactor - 1} \right) \VarInitCapacity \FluteLess \frac {\VarGrowthFactor{n}} {\VarGrowthFactor - 1}\\
\left( \frac{\VarGrowthFactor}{\VarGrowthFactor - 1} \right) n &\FluteLeq \FuncWrites(n) \FluteLess \left( \frac{2\VarGrowthFactor - 1}{\VarGrowthFactor - 1} \right) n
\end{align*}
\HdrSpaceComplex

I wish to find the space allocated when $n$ items are appended to a collection. I call this quantity the \textbf{space cost}, denote it $\FuncSpace(n)$, and define it as the total length of buffers allocated by $n$ appends. Now, I derive a formula for $\FuncSpace(n)$.

First, from the definition of $\VarList.\FieldCapacity$, note that a dynamic array's capacity is the length of the buffer it stores its items in. Then a buffer of length $c$ is allocated at some point if and only if $c \in \VarCapacitySeq$. Then the total length of those buffers is
\begin{align*}
\FuncSpace(n) &= \sum_i {\VarCapacitySeq_i}\\
&= \VarInitCapacity + \VarGrowthFactor\VarInitCapacity + \VarGrowthFactor^2\VarInitCapacity + \ldots + \VarGrowthFactor^\VarUseful\VarInitCapacity\\
&= \left( \frac{\VarGrowthFactor^{\VarUseful + 1} - 1}{\VarGrowthFactor - 1} \right) \VarInitCapacity
\end{align*}
Using Lemma \ref{lem:ToVarUsefulPowerBounds} again, I asymptotically bound $\FuncSpace(n)$:
\begin{align*}
\frac n \VarInitCapacity &\FluteLeq \VarGrowthFactor^\VarUseful \FluteLess \frac {\VarGrowthFactor{n}} \VarInitCapacity\\
\frac {\VarGrowthFactor n} \VarInitCapacity &\FluteLeq \VarGrowthFactor^{\VarUseful + 1} \FluteLess \frac {\VarGrowthFactor^2 n} \VarInitCapacity\\
\frac {\VarGrowthFactor n} \VarInitCapacity &\FluteLeq \VarGrowthFactor^{\VarUseful + 1} - 1 \FluteLess \frac {\VarGrowthFactor^2 n} \VarInitCapacity\\
\left( \frac{\VarGrowthFactor}{\VarGrowthFactor - 1} \right) n &\FluteLeq \FuncSpace(n) \FluteLess \left( \frac{\VarGrowthFactor^2}{\VarGrowthFactor - 1} \right) n
\end{align*}