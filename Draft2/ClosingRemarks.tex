In this paper, I developed the concept of growth arrays and showed that they are a viable replacement for dynamic arrays. I implemented several common dynamic array operations for growth arrays, and showed that for appending, growth arrays outperform dynamic ones both time- and memory-wise. I demonstrated that although growth arrays are slower at random access than dynamic arrays, it is still possible to index them in constant time. Finally, I showed that they perform comparably to dynamic arrays for iteration and conversion to raw arrays.

I will end this paper with the following open-ended questions, which may serve as inspiration for further research:

\begin{itemize}
	\item Can growth arrays' growth algorithm bring performance improvements to other array-based structures, such as circular queues or array-based stacks?
	\item Is there an efficient, constant time random access algorithm for growth arrays that works for all values of $\VarGrowthFactor$ and $\VarInitCapacity$?
	\item To what extent does growth arrays' fragmented structure result in performance losses due to poorer data locality? Can memory allocators offset these losses by allocating buffers close to each other?
	\item How can algorithms such as binary search and sorting be best implemented for growth arrays? Since growth arrays with $\VarGrowthFactor = 2$ have an evenly-divided structure, is there a better binary search algorithm for them?
\end{itemize}