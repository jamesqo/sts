\documentclass{article}
\usepackage{amsmath}

\newcommand{\descriptn}{\textbf{Description}}
\newcommand{\dynarrayimpl}{\textbf{Dynamic array implementation}}
\newcommand{\funarrayimpl}{\textbf{Funnel array implementation}}
\newcommand{\tcomplex}{\textbf{Time complexity}}
\newcommand{\scomplex}{\textbf{Space complexity}}
\newcommand{\tcomplexcmp}{\textbf{Time complexity comparison}}
\newcommand{\scomplexcmp}{\textbf{Space complexity comparison}}

\newcommand{\timefn}{T}
\newcommand{\spacefn}{S}
\newcommand{\nwritesfn}{W}

\newcommand{\timenewfn}{\timefn'}
\newcommand{\spacenewfn}{\spacefn'}
\newcommand{\nwritesnewfn}{\nwritesfn'}

\newcommand{\nwritesresizefn}{W_R}

\newcommand{\timeratio}{r_\timefn}
\newcommand{\spaceratio}{r_\spacefn}

\newcommand{\bigo}{O}
\newcommand{\biggo}{P}
\newcommand{\varnitems}{n}
\newcommand{\indexertime}{I}

\newcommand{\listname}{L}

\newcommand{\initcapacity}{C_f}
\newcommand{\growthfactor}{G}
\newcommand{\capacityfn}{C}

\newcommand{\initcapacitynew}{\initcapacity'}
\newcommand{\growthfactornew}{\growthfactor'}

\setlength{\parskip}{1em}

\begin{document}
	\begin{abstract}
	\end{abstract}

	\section{Introduction}

	In imperative languages, the dynamic array is the most common data structure used by programs. People often want to add multiple items to a collection and then iterate it, which dynamic arrays make simple and efficient. But can it be said they have the \textit{most} efficient algorithm for this pattern?

In this paper, I introduce an alternative data structure to the dynamic array, called the \textbf{growth array}. It is more efficient than the dynamic array at appending large numbers of items. This is due to how it 'grows' once it cannot fit more items in its buffer.

When dynamic arrays run out of space, they allocate a new buffer, copy the contents of the old one into it, and throw the old one away. However, growth arrays are less wasteful. Instead of throwing away the filled buffer, they keep it a part of the data structure. The new buffer they allocate represents a continuation of the items from the old buffer. For example, if the old buffer contained items $0 - 31$, the new buffer would contain items $32$ and beyond. Because of this, growth arrays allocate less memory to store the same number of items, and they do not need to copy items from the old buffer to the new one.

Growth arrays have caveats, however. They perform no better, or slightly worse, than dynamic arrays for operations other than appending. Random access, in particular, involves many more instructions than it does for dynamic arrays. Also, since growth arrays are not contiguous in memory, they may have poorer locality than dynamic arrays, and cannot be passed to external code that accepts contiguous buffers.

It is worth mentioning that if the size of the data is known in advance, both dynamic and growth arrays are completely unnecessary. A raw array could simply be allocated with the known size, and items could be appended to it just as quickly. Thus, growth arrays are only beneficial for cases where the amount of data to be appended is unknown, but is expected to be large.
	
	\section{Fields and Properties}
	
	I define some \textbf{fields} and \textbf{properties} (constant time methods that do not change state) for both types of arrays, so I may use them later to implement non-trivial methods.
	
	\section{Common Operations}
	
	In this section, I implement and analyze common operations for dynamic and funnel arrays. I compare both implementations' time complexities, and space complexities if the operation allocates memory.
	
	The following mathematical definitions will be used while analyzing time and space complexity:
	
	\begin{description}
		\item[$\biggo(f(n))$] This is an alternative to big-O notation that I will name "big-P notation". It is similar to big-O, but it preserves the coefficient of the fastest-growing term in $f(n)$. For example, $\bigo(2n) = \bigo(n)$, but $\biggo(2n) \neq \biggo(n)$. This makes it possible to compare two time complexities, if their ratio tends to a constant for large $n$.\\
		More formally, $\biggo(f(n)) = \biggo(g(n))$ iff $$\lim_{n \to \infty} {\frac{f(n)}{g(n)}} = 1$$.
	\end{description}
	
	\subsection{Adding}
	
	\descriptn
	
	Adding is the most common operation done on dynamic arrays\footnote{}. Funnel arrays improve the performance of adding in two ways: by allocating less memory, and reducing the amount of copying.
	
	\HdrDynArrayImpl

I will implement appending for dynamic arrays first. Let $\VarList$ be a dynamic array. The following definitions are used in the code:

\begin{description}
	\item[initial capacity] Denoted by $\VarInitCapacity$. The capacity of a dynamic a array when one item is appended to it.\\
	{\HdrNote} The capacity of any dynamic array with size zero should be zero.
	\item[growth factor] Denoted by $\VarGrowthFactor$. The factor by which the current capacity is multiplied to get the new capacity when $\VarList$ grows.
\end{description}

\begin{algorithm}
	\begin{algorithmic}[1]
		\Procedure{$\FuncAppend$}{$L, item$}
			\If{$\VarList.\FieldFull$}
				\State $\VarList.\FuncGrow()$
			\EndIf
			
			\State $\VarList.\FieldHead[\VarList.\FieldHeadSize] \gets item$
			\State $\VarList.\FieldSize \gets \VarList.\FieldSize + 1$
		\EndProcedure
		\Statex
		\Procedure{$\FuncGrow$}{$L$}
			\If{$\VarList.\FieldEmpty$}
				\State $\VarList.\FieldBuffer \gets \FuncNewArray(\VarInitCapacity)$
			\Else
				\State $new\ buf \gets \FuncNewArray(\VarList.\FieldSize \times \VarGrowthFactor)$
				\State $\FuncArrayCopy(\VarList.\FieldBuffer, new\ buf, \VarList.\FieldSize)$
				\State $\VarList.\FieldBuffer \gets new\ buf$
			\EndIf
		\EndProcedure
	\end{algorithmic}
\end{algorithm}

\HdrTimeComplex

Before I analyze the time complexity of $\FuncAppend$, I consider a different method for measuring its cost. Suppose I start with an empty collection and $n$ elements are appended. How many times is an element stored in an array? I will term the answer to this question the \textbf{write cost} of $n$ appends, and denote it $\FuncWrites(n)$.

In the code for $\FuncAppend$, one array store is performed unconditionally, so it is apparent that $\FuncWrites(n) \geq n$ after $n$ appends. However, $\FuncGrow$ also does some writing, so in order to find a precise formula for $\FuncWrites(n)$, I need to analyze when $\FuncGrow$ is called. To do this, I use the following lemma:

\begin{lemma}
\label{lem:CapacitySequence}
	Let $\VarList$ by a dynamic array. Let its \textbf{capacity sequence}, $\VarCapacitySequence$, be the capacities $\VarList$ takes on as $n$ items are appended. Then
	
	$$
	\VarCapacitySequence = 0,\ \VarInitCapacity,\ \VarGrowthFactor\VarInitCapacity,\ \VarGrowthFactor^2\VarInitCapacity,\ \ldots\ \VarGrowthFactor^{\lceil \log_{\VarGrowthFactor} n - \log_{\VarGrowthFactor} \VarInitCapacity \rceil}\VarInitCapacity
	$$
\end{lemma}

% Bookmark

\begin{proof}
	The capacity starts out at zero and becomes $\VarInitCapacity$ once one element is appended. From then on, it can only grow by a factor of $\VarGrowthFactor$ via $\FuncGrow$, so if $\VarGrowthFactor^i\VarInitCapacity$ is the current capacity then $\VarGrowthFactor^{i + 1}\VarInitCapacity$ must be the next capacity. By induction, the rest of the sequence is $$\left\{ \VarGrowthFactor^i\VarInitCapacity \right\}_{i = 1}^k$$ for some $k$. Since the capacity is only as big as is needs to be, namely greater than or equal to $n$, $k$ is the smallest integer for which $\VarGrowthFactor^k\VarInitCapacity \geq n$. Using this property, I now derive $k$.
	
	\begin{align*}
	\VarGrowthFactor^k\VarInitCapacity &\geq n\\
	\VarGrowthFactor^k &\geq \frac{n}{\VarInitCapacity}\\
	k &\geq \log_{\VarGrowthFactor} n - \log_{\VarGrowthFactor} \VarInitCapacity\\
	k &\geq \log_{\VarGrowthFactor} n - \log_{\VarGrowthFactor} \VarInitCapacity > k - 1\\
	k &= \big\lceil \log_{\VarGrowthFactor} n - \log_{\VarGrowthFactor} \VarInitCapacity \big\rceil
	\end{align*}
	
	Then the final term in the sequence is $\VarGrowthFactor^{\lceil \log_{\VarGrowthFactor} n - \log_{\VarGrowthFactor} \VarInitCapacity \rceil}\VarInitCapacity$, completing the proof.
\end{proof}

[make lemma]
Now, consider that to grow by $\VarGrowthFactor$, $\FuncGrow$ must be called. So $\VarGrowthFactor^k\VarInitCapacity$ being present in $S_C$ means $\FuncGrow$ was called at size $\VarGrowthFactor^{k - 1}\VarInitCapacity$ for $k \geq 1$. Then $\FuncGrow$ was called at the sizes $$S_R = 0,\ \VarInitCapacity,\ \VarGrowthFactor\VarInitCapacity,\ \VarGrowthFactor^2\VarInitCapacity,\ \ldots\ \VarGrowthFactor^{\lceil \log_{\VarGrowthFactor} n - \log_{\VarGrowthFactor} \VarInitCapacity \rceil - 1}\VarInitCapacity$$

Each time $\FuncGrow$ is called, $i$ items are copied where $i$ is the current size of the dynamic array. Then the total number of items copied for $n$ appends is

\begin{align*}
0 + \VarInitCapacity + \VarGrowthFactor\VarInitCapacity + \ldots + \VarGrowthFactor^{\lceil \log_{\VarGrowthFactor} n - \log_{\VarGrowthFactor} \VarInitCapacity \rceil - 1}\VarInitCapacity &= \left( \frac{\VarGrowthFactor^{\lceil \log_{\VarGrowthFactor} n - \log_{\VarGrowthFactor} \VarInitCapacity \rceil} - 1}{\VarGrowthFactor - 1} \right) \VarInitCapacity
\end{align*}

Finally, counting the writes made for every item in $\FuncAppend$, I finally find an explicit formula for $\FuncWrites(n)$.

[ref1]
$$
\FuncWrites(n) = n + \left( \frac{\VarGrowthFactor^{\lceil \log_{\VarGrowthFactor} n - \log_{\VarGrowthFactor} \VarInitCapacity \rceil} - 1}{\VarGrowthFactor - 1} \right) \VarInitCapacity
$$

[ref2]
To make this expression easier to work with, I will write it in big-P notation to approximate it for large $n$. First I note that $\log_{\VarGrowthFactor} k \leq \lceil \log_{\VarGrowthFactor} k \rceil < \left( \log_{\VarGrowthFactor} k \right) + 1$ for any $k > 0$, so $k \leq \VarGrowthFactor^{\lceil \log_{\VarGrowthFactor} k \rceil} < \VarGrowthFactor{k}$. Then

\begin{align*}
\FuncWrites(n) &= n + \left( \frac{\VarGrowthFactor^{\big\lceil \log_{\VarGrowthFactor} \left( \frac {n} {\VarInitCapacity} \right) \big\rceil} - 1}{\VarGrowthFactor - 1} \right) \VarInitCapacity\\
n + \left( \frac{\frac{n}{\VarInitCapacity} - 1}{\VarGrowthFactor - 1} \right) \VarInitCapacity \leq \FuncWrites(n) &< n + \left( \frac{\VarGrowthFactor\frac{n}{\VarInitCapacity} - 1}{\VarGrowthFactor - 1} \right) \VarInitCapacity\\
n + \left( \frac{n - \VarInitCapacity}{\VarGrowthFactor - 1} \right) \leq \FuncWrites(n) &< n + \left( \frac{\VarGrowthFactor{n} - \VarInitCapacity}{\VarGrowthFactor - 1} \right)\\
P\left( n + \left( \frac{n - \VarInitCapacity}{\VarGrowthFactor - 1} \right) \right) \leq P\left( \FuncWrites(n) \right) &< P\left( n + \left( \frac{\VarGrowthFactor{n} - \VarInitCapacity}{\VarGrowthFactor - 1} \right) \right)\\
P\left( \left( \frac{\VarGrowthFactor}{\VarGrowthFactor - 1} \right) n \right) \leq P\left( \FuncWrites(n) \right) &< P\left( \left( \frac{2\VarGrowthFactor - 1}{\VarGrowthFactor - 1} \right) n \right)
\end{align*}

\HdrSpaceComplex

I wish to find the space allocated while appending $n$ items to a dynamic array. I denote this quantity $\FuncSpace(n)$, and define it as the total length of all arrays allocated during $n$ $\FuncAppend$ calls. I now derive a formula for $\FuncSpace(n)$.

From \ref{lem:CapacitySequence} I know the exact capacities $\VarList$ takes on as $n$ items are appended. Dynamic arrays also have the property that $i \in S_C$ iff an array of length $i$ was allocated. Then the total length of arrays allocated is

\begin{align*}
\FuncSpace(n) &= \sum_{i \in S_C} i\\
&= 0 + \VarInitCapacity + \VarGrowthFactor\VarInitCapacity + \VarGrowthFactor^2\VarInitCapacity + \ldots + \VarGrowthFactor^{\lceil \log_{\VarGrowthFactor} n - \log_{\VarGrowthFactor} \VarInitCapacity \rceil}\VarInitCapacity\\
&= \left( \frac{\VarGrowthFactor^{\lceil \log_{\VarGrowthFactor} n - \log_{\VarGrowthFactor} \VarInitCapacity \rceil + 1} - 1}{\VarGrowthFactor - 1} \right) \VarInitCapacity
\end{align*}

Using the identity in [ref2] again, I bound its big-P complexity:

\begin{align*}
\left( \frac{\VarGrowthFactor \frac{n}{\VarInitCapacity} - 1}{\VarGrowthFactor - 1} \right) \VarInitCapacity \leq \FuncSpace(n) &< \left( \frac{\VarGrowthFactor^2 \frac{n}{\VarInitCapacity} - 1}{\VarGrowthFactor - 1} \right) \VarInitCapacity\\
P\left( \left( \frac{\VarGrowthFactor \frac{n}{\VarInitCapacity} - 1}{\VarGrowthFactor - 1} \right) \VarInitCapacity \right) \leq P\left( \FuncSpace(n) \right) &< P\left( \left( \frac{\VarGrowthFactor^2 \frac{n}{\VarInitCapacity} - 1}{\VarGrowthFactor - 1} \right) \VarInitCapacity \right)\\
P\left( \left( \frac{\VarGrowthFactor}{\VarGrowthFactor - 1} \right) n \right) \leq P\left( \FuncSpace(n) \right) &< P\left( \left( \frac{\VarGrowthFactor^2}{\VarGrowthFactor - 1} \right) n \right)
\end{align*}
	
	\HdrGrowthArrayImpl

\HdrTimeComplex

I start off again by finding the write cost for $n$ items. Lemma \ref{lem:CapacitySeq} still holds, since growth arrays satisfy the properties used by that proof. In particular, although growth arrays use a different growth algorithm than dynamic arrays, the following claim is still true:

\begin{lemma}
\label{lem:GrowthArraysGrowthFactor}
	The capacity of a growth array grows by the constant factor $\VarGrowthFactor$.
\end{lemma}

\begin{proof}
	I prove that the $\FuncGrow$ algorithm enforces this using induction. I induct on the number of times $\FuncGrow$ is called, $k$, showing that for all natural numbers $k$, $\FuncGrow$ behaves correctly when called the $k$th time. I will let $c_i$ and $c_f$ denote the initial/final capacities and $h_i$ and $h_f$ denote the initial/final head capacities for the $k$th call, respectively.

	For $k = 1$, $c_i = \VarInitCapacity$. I wish to show that $c_f = \VarGrowthFactor\VarInitCapacity$. This happens if and only if the next buffer has size $\Delta c = (\VarGrowthFactor - 1)\VarInitCapacity$, which the algorithm ensures.
	
	For $k > 1$, by induction $c_i = \text{previous }c_f = \VarGrowthFactor^{k - 2}\VarInitCapacity$, and $h_i = \text{previous }h_f = (\VarGrowthFactor^{k - 2} - \VarGrowthFactor^{k - 3})\VarInitCapacity$. I wish to show $c_f = \VarGrowthFactor^{k - 1}\VarInitCapacity$ and $h_f = (\VarGrowthFactor^{k - 1} - \VarGrowthFactor^{k - 2})\VarInitCapacity$. Because $k > 1$, the algorithm will calculate $h_f$ as $\VarGrowthFactor$ times $h_i$. Then
	\begin{align*}
	h_f &= \VarGrowthFactor h_i = \VarGrowthFactor (\VarGrowthFactor^{k - 2} - \VarGrowthFactor^{k - 3})\VarInitCapacity = (\VarGrowthFactor^{k - 1} - \VarGrowthFactor^{k - 2})\VarInitCapacity
	\end{align*}
	and
	\begin{align*}
	c_f &= c_i + h_f = \VarGrowthFactor^{k - 2}\VarInitCapacity + (\VarGrowthFactor^{k - 1} - \VarGrowthFactor^{k - 2})\VarInitCapacity = \VarGrowthFactor^{k - 1}\VarInitCapacity
	\end{align*}
	as desired.
\end{proof}

Since Lemma \ref{lem:GrowthArraysGrowthFactor} has been proven, Lemma \ref{lem:CapacitySeq} and all results based on it must also hold true for growth arrays. Now, I am ready to find the write cost of $\FuncGrow$. Unlike dynamic arrays, $\FuncGrow$ does not make $\VarGrowSeq_i$ writes when the current size is $\VarGrowSeq_i$. In fact, $\FuncGrow$ does not copy \textit{any} items supplied by the user. Writes are only made when a buffer is appended to the tail, since the tail is a dynamic array.

Let $\FuncWritesByGrow(n)$ denote the total number of writes made by $\FuncGrow$, and let $\VarNumItemsTail$ be the size of the tail. Since Corollary \ref{coro:GrowthSeq} also holds true for growth arrays, $\FuncGrow$ is called $|\VarGrowSeq|$ times. A buffer is appended to the tail each time $\FuncGrow$ is called. Thus, the tail's size is
\begin{align*}
\VarNumItemsTail &= |\VarGrowSeq| = |\VarCapacitySeq| - 1 = \VarUseful
\end{align*}
Then the formula for $\FuncWritesByGrow(n)$ is simply $\FuncWritesTail(\VarUseful)$, where $\FuncWritesTail$ denotes the tail's write cost function, that is, the write cost function for dynamic arrays. Finally, adding the writes made by $\FuncAppend$, the formula for $\FuncWrites(n)$ is
\begin{align*}
\FuncWrites(n) = n + \FuncWritesTail(\VarUseful)
\end{align*}
Now, I approximate $\FuncWrites(n)$ using $\sim$. To do this, I will derive the big-O complexity of $\VarUseful$.

\begin{lemma}
\label{lem:VarUsefulIsOLogN}
	$\VarUseful = O(\log n)$.
\end{lemma}

\begin{proof}
	From \ref{lem:CapacitySeq}, $\VarUseful = \max(\ExprMessy, 0)$. As mentioned in Lemma \ref{lem:ToVarUsefulPowerBounds}, for sufficiently large $n$, $\VarUseful = \ExprMessy$. Then
	\begin{align*}
	\VarUseful \sim \ExprMessy \sim \left( \ExprMessyInner \right) \sim \log_{\VarGrowthFactor} n
	\end{align*}
	By [lemma], $O(\VarUseful) = O(\log_{\VarGrowthFactor} n) = O(\log n)$ as desired.
\end{proof}

Now when $\FuncWrites(n)$ is approximated with $\sim$, the $\FuncWritesTail(\VarUseful)$ term disappears:
\begin{align*}
\FuncWrites(n) &= n + \FuncWritesTail(\VarUseful) = n + O(\VarUseful) = n + O(\log n) \sim n
\end{align*}

\HdrSpaceComplex

Unlike dynamic arrays, growth arrays never throw away buffers. This means that if the current capacity is $c$, then the total length of buffers allocated to store items is also $c$. Typically, however, $\FuncSpace(n) > c$. This is because growth arrays not only store items in buffers, they also store \textbf{references} (or \textbf{pointers}) to the buffers holding the items, in the tail. Thus, the space the tail allocates must also be considered.

We established in [lemma] that $\VarNumItemsTail = \VarUseful$. Then since $\VarCapacitySeq_\VarUseful$ is the space needed to hold items, and $\FuncSpaceTail(\VarUseful)$ is the space the tail needs to hold references, the formula for $\FuncSpace(n)$ is
\begin{align*}
\FuncSpace(n) = \VarCapacitySeq_\VarUseful + \FuncSpaceTail(\VarUseful)
\end{align*}
I now wish to approximate this using $\sim$. First, note that since $\VarCapacitySeq_\VarUseful = \VarGrowthFactor^\VarUseful\VarInitCapacity$, it follows from Lemma \ref{lem:ToVarUsefulPowerBounds} that $n \FluteLeq \VarCapacitySeq_\VarUseful \FluteLess \VarGrowthFactor{n}$. Since $n$ and $\VarGrowthFactor{n}$ are both $O(n)$, [the squeeze theorem] implies that $\VarCapacitySeq_\VarUseful = O(n)$.

Now, consider the ratio $\lim_{n \to \infty} \dfrac {\FuncSpaceTail(\VarUseful)} {\VarCapacitySeq_\VarUseful} = \dfrac {O(\log n)} {O(n)} = O \left( \dfrac{\log n}{n} \right) = 0$. Because it is $0$, from Theorem \ref{thm:RemovesLowerOrderTerms} one can conclude
\begin{align*}
\FuncSpace(n) = \VarCapacitySeq_\VarUseful + \FuncSpaceTail(\VarUseful) \sim \VarCapacitySeq_\VarUseful
\end{align*}
Substituting $\FuncSpace(n)$ for $\VarCapacitySeq_\VarUseful$ in the previous inequality,
\begin{align*}
n \FluteLeq \FuncSpace(n) \FluteLess \VarGrowthFactor{n}
\end{align*}
	
	\HdrTimeComplexCmp

In this section, $\FuncWritesDynamicArray$ and $\FuncWritesGrowthArray$ will denote the write cost functions for dynamic and growth arrays, respectively. I derived earlier that
\begin{align*}
\left( \frac{\VarGrowthFactor}{\VarGrowthFactor - 1} \right) n \FluteLeq \FuncWritesDynamicArray(n) &\FluteLess \left( \frac{2\VarGrowthFactor - 1}{\VarGrowthFactor - 1} \right) n
\end{align*}
From Theorem \ref{thm:BothSidesInequality}, I may divide all sides of this inequality by $n$ to receive
\begin{align*}
\frac{\VarGrowthFactor}{\VarGrowthFactor - 1} \FluteLeq \frac {\FuncWritesDynamicArray(n)} n &\FluteLess \frac{2\VarGrowthFactor - 1}{\VarGrowthFactor - 1}
\end{align*}
It was shown earlier that $\FuncWritesGrowthArray(n) \sim n$. From Theorem \ref{thm:MergesOverOps}, $\dfrac {\FuncWritesDynamicArray(n)} n \sim \dfrac{\FuncWritesDynamicArray(n)}{\FuncWritesGrowthArray(n)}$. From Theorem \ref{thm:InterchangeableInInequality}, it follows that
\begin{align*}
\frac{\VarGrowthFactor}{\VarGrowthFactor - 1} \FluteLeq \frac {\FuncWritesDynamicArray(n)} {\FuncWritesGrowthArray(n)} &\FluteLess \frac{2\VarGrowthFactor - 1}{\VarGrowthFactor - 1}
\end{align*}
\HdrSpaceComplexCmp

Let $\FuncSpaceDynamicArray(n)$ and $\FuncSpaceGrowthArray(n)$ be the space cost functions for dynamic and growth arrays, respectively. Recall that
\begin{align*}
\FuncSpaceDynamicArray(n) = \left( \frac{\VarGrowthFactor^{\VarUseful + 1} - 1}{\VarGrowthFactor - 1} \right) \VarInitCapacity
\end{align*}
and that $\FuncSpaceGrowthArray(n) \sim \VarCapacitySeq_\VarUseful = \VarGrowthFactor^\VarUseful\VarInitCapacity$. Taking the ratio of $\FuncSpaceDynamicArray(n)$ to $\FuncSpaceGrowthArray(n)$, I receive
\begin{align*}
\frac{\FuncSpaceDynamicArray(n)}{\FuncSpaceGrowthArray(n)} &\sim \frac{1}{\VarGrowthFactor^\VarUseful\VarInitCapacity} \cdot \left( \frac{\VarGrowthFactor^{\VarUseful + 1} - 1}{\VarGrowthFactor - 1} \right) \VarInitCapacity &&\text{(Theorem \ref{thm:MergesOverOps})}\\
&= \frac{\VarGrowthFactor - 1 / {\VarGrowthFactor^\VarUseful}}{\VarGrowthFactor - 1}\\
&\sim \frac{\VarGrowthFactor}{\VarGrowthFactor - 1} &&\text{(Theorem \ref{thm:RemovesLowerOrderTerms})}
\end{align*}
	
	\subsection{Indexing}
	
	\descriptn
	
	Indexing is another very common operation on a list. I will call methods that get or set an item at a specified index \textbf{get} and \textbf{set indexers}, respectively.
	
	\dynarrayimpl
	
	\tcomplex
	
	$\timefn(\varnitems) = \bigo(1)$
	
	% <funnel array implementation>
	
	\tcomplex
	
	$\timenewfn(\varnitems) = \bigo(1)$
	
	\tcomplexcmp
	
	\subsection{Iterating}
	
	\descriptn
	
	\textbf{Iteration} of a list is the process of performing some action on each of its elements.
	
	\dynarrayimpl
	
	\tcomplex
	
	Ignoring the arbitrary code run after getting each element, the time complexity of this method is $\bigo(\varnitems)$.
	
	\funarrayimpl
	
	\tcomplex
	
	$\timenewfn(\varnitems) = \bigo(\varnitems)$
	
	\tcomplexcmp
	
	\subsection{Copying to an array}
	
	\descriptn
	
	Users often want to take list structures, such as dynamic arrays, and convert them into plain arrays. There are multiple reasons why someone would want to do this after they are done adding to the list:
	
	\begin{itemize}
		\item Plain arrays hold on to exactly the amount of memory needed to hold their elements. However, dynamic and funnel arrays allocate more space than necessary to optimize adding new items.
		\item The user wants to call a function in third-party code that takes a plain array as an argument.
		\item
		\item Plain arrays are contiguous, while funnel arrays are fragmented and have worse locality.
		\item The indexer of funnel arrays is several times slower than that of plain arrays, whether the $\bigo(1)$ or $\bigo(\log \varnitems)$ implementation is chosen.
	\end{itemize}
	
	\dynarrayimpl
	
	\tcomplex
	
	\funarrayimpl
	
	\tcomplex
	
	\section{Other Operations}
	
	\subsection{Inserting}
	
	\descriptn
	
	\textbf{Inserting} an element places it at a specified index within the list, and increments the list's count. If the index equals the size of the list, the effect is the same as adding the element. Otherwise, the elements at indices greater than or equal to the specified index are moved to the next index, then the element is placed at the specified index.
	
	\subsection{Deleting}
	
	\descriptn
	
	\textbf{Deleting} the element at a specified index moves the elements at indices greater than the specified index to the previous index. The count of the list is decremented.
	
	\subsection{Searching}
	
	\descriptn
	
	\textbf{Searching} for an element returns the first or last index within the list where the item can be found. If the user knows the items of the lists are sorted, \textbf{binary search} can be used.
	
	\subsection{In-place sorting}
	
	\descriptn
	
	Given a strict ordering $<$, I say a list $\listname$ is \textbf{sorted} by $<$ iff $\left(a, b \in \listname \land a < b\right) \leftrightarrow I_M(a) < I_m(b)$. $I_M(a)$ is the last (maximum) index of $a$ in $L$, and $I_m(b)$ is the first (minimum) index of $b$ in $L$. Note that the use of $<$ on the right-hand side compares integers and not elements of $L$, so this is not a recursive definition.
	
	\section{Implementations}
	
	\section{Benchmarks}
	
	\section{Closing Remarks}

\end{document}
