\HdrGrowthArrayImpl

The following algorithm converts a growth array to a raw array:

\begin{algorithm}[H]
	\caption{Converting to a raw array \TextGrowthArray}
	\begin{algorithmic}
		\Function{$\FuncToArray$}{$\VarList$}
		\State $\LclRawArray \gets \FuncNewArray(\VarList.\FieldSize)$
		\For{$\LclBuffer \TextIn \VarList.\FieldTail$}
			\State $\FuncArrayCopy(\LclBuffer,\ \LclRawArray,\ \LclBuffer.\FieldLength)$
		\EndFor
		\State $\FuncArrayCopy(\VarList.\FieldHead,\ \LclRawArray,\ \VarList.\FieldHeadSize)$
		\State \Return $\LclRawArray$
		\EndFunction
	\end{algorithmic}
\end{algorithm}

This algorithm copies $n$ items, so its time complexity is $O(n)$. Using the same argument as in Section \ref{subsec:Iterating}, the cost of `interruptions' from switching buffers is minuscule. Ignoring $\FuncNewArray$ since it is non-deterministic, the only substantial work done by this function is the multiple $\FuncArrayCopy$ calls. They constitute roughly the same work done by dynamic arrays' single $\FuncArrayCopy$ call. Thus, $\FuncToArray$ should take roughly as long for growth arrays as it does for dynamic arrays.