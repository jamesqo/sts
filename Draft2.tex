\documentclass{article}
\usepackage{amsmath}

\newcommand{\tcomplex}{\textbf{Time complexity}}
\newcommand{\scomplex}{\textbf{Space complexity}}

\newcommand{\timefn}{T}
\newcommand{\spacefn}{S}
\newcommand{\nwritesfn}{W}

\newcommand{\bigo}{O}
\newcommand{\biggo}{P}
\newcommand{\varnitems}{n}

\setlength{\parskip}{1em}

\begin{document}
	\begin{abstract}
	\end{abstract}

	\section{Introduction}

	\section{Definitions}

	\section{Predefined Functions}
	
	\section{Fields and Properties}
	
	In this section, I define \textbf{fields} and \textbf{properties} on dynamic arrays and resize lists. These will be used later to implement operations on them in pseudocode.
	
	\section{Common Resize List Operations}
	
	In this section, I implement common operations for dynamic arrays, then resize lists, in pseudocode. Then, I analyze and compare the time complexities of the implementations. If the operation allocates memory, I also compare their space complexities.
	
	\subsection{Adding}
	
	Adding is the most common operation done on a dynamic array\footnote{}. Resize lists improve the performance of adding in two ways: by allocating less memory, and reducing the amount of copying.
	
	% <dynamic array implementation>
	
	\tcomplex
	
	\begin{align*}
	\nwritesfn(\varnitems) = \biggo(2\varnitems)
	\end{align*}
	
	\scomplex
	
	\begin{align*}
	\spacefn(\varnitems) = \biggo(2^{\lceil \log_2 \varnitems \rceil + 1}) = \bigo(\varnitems)
	\end{align*}
	
	% <resize list implementation>
	
	\tcomplex
	
	\begin{align*}
	\nwritesfn(\varnitems) = \biggo(\varnitems)
	\end{align*}
	
	\scomplex
	
	\begin{align*}
	\spacefn(\varnitems) = \biggo(2^{\lceil \log_2 \varnitems \rceil}) = \bigo(\varnitems)
	\end{align*}
	
	\subsection{Indexing}
	
	\subsection{Iterating}
	
	\subsection{Copying to an array}
	
	\section{Other Resize List Operations}
	
	\subsection{Inserting}
	
	\subsection{Deleting}
	
	\subsection{Searching}
	
	\subsection{In-place sorting}
	
	\section{Implementations}
	
	\section{Benchmarks}
	
	\section{Closing Remarks}

\end{document}
